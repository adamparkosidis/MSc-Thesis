\chapter{Binary Star Systems}

The evolution of a star in isolation could be summarized as: Stars contract because they are hot and lose energy through radiation (Virial Theorem), while nuclear burning cycles serve as long-lived but transitory interruptions to a star's (or at least its core's) inevitable contraction due to gravity. Despite that seemingly straightforward picture, stars tend form in multiple systems (\cref{fig:stellar_companions}). Binaries particularly, which consist of two stars that orbit around a shared center of mass and are gravitationally connected to each other, are of immense importance. By nature,  they reveal more about themselves, particularly their masses and diameters, than other astronomical objects. This is especially true for eclipsing binaries \citep{prvsa2016physics}, which provide us with direct knowledge regarding spatial relations inside the source. Furthermore, close binaries are unique cosmic laboratories providing useful insights regarding different physical process. For example, gravitational wave emission is widely studied in binary mergers where both stars are compact objects \citep{cutler1994gravitational,abbott2017gw170608,abbott2019gwtc}, accretion as a power source in X-ray binaries \citep{lewin1997x,reig2011x}, while stellar interactions such as mass transfer, angular momentum exchange and tidal friction in close binaries {\it citation to be}. 

There are two fundamental reasons why many binaries transfer matter at some point in their evolution:

\begin{itemize}
    \item During its development, one of the stars in a binary system may expand in radius or contract in binary separation to the point where the companion's gravitational force can remove the outer layers of its envelope (Roche lobe overflow).
    \item At some point in its history, one of the stars may release most of its mass in the form of a stellar wind; part of this material will be gravitationally trapped by the companion (stellar wind accretion).
\end{itemize}

Despite the additional complexity that these stellar interactions provide to the evolution of these systems, they offer the necessary base for discussing stellar interactions in triple systems. This chapter is mainly focused on MS-MS interacting binaries via the first method


%Binaries' orbital periods and distances vary greatly. Some systems are so close that the stars' surfaces are virtually touching and can interchange material, namely contact binaries {\it(citation to be)}. Others may be separated by thousands of astronomical units and have orbital periods of hundreds of years. 

\section{Roche Lobe Overflow}

The evolution of such system can be described by the masses of the binary components, $M_1$ and $M_2$, their orbital separation,$\alpha$, and the eccentricity, $e$, of their orbit \citep{postnov2014evolution,sana2012binary,toonen2014popcorn}.

