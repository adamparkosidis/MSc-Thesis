\section{Target system: $\xi$ Tau}

$\xi$ Tau is a hierarchical triple system with orbital parameters that are relatively well constrained. The evolution of the system was initially examined by \cite{de2014evolution}, where they  concluded that it is likely to develop Roche lobe overflow near the end of the first ascend, roughly halfway to type B mass transfer \citep{kippenhahn1967entwicklung}.

To further restrict the system's orbital characteristics, \cite{nemravova2016xitauri} combined a large series of spectroscopic photometric (including space-borne) observations with long-baseline optical and infrared spectro-interferometric observations. They show, using perturbation theory, that prominent secular and periodic dynamical effects may be explained by a quadrupole interaction. Despite this, the orbital parameters of the third orbit remain unclear due to the low relative brightness of the most distant star.

I use the orbital parameters given in \cite{2010yCat..73890925T} in my study. The rationale for this is so that I can compare my results to \cite{de2014evolution}. The orbital parameters of $\xi$ Tau are given in \cref{tab:system_orbit_param}.
\begin{table}[H]
    \centering
    \begin{tabular}{|c c c c c c c c|}
       Name & M$_1$ (M$_{\odot}$) & M$_2$ (M$_{\odot}$) &
       M$_3$ (M$_{\odot}$) & $P_{in}$ (day) &
       $P_{out}$ (day) & $\epsilon_{in}$ &
       $\epsilon_{out}$ \\
       \hline
       HD 21364 & 3.2 & 3.1 & 5.5 & 7.15 & 145.5 & 0.0 & 0.15
    \end{tabular}
    \caption{ Orbital parameters of $\xi$ Tau system}
    \label{tab:system_orbit_param}
\end{table}
In their analysis, \cite{nemravova2016xitauri} constraint  the mutual inclination between the two orbits, concluded that they are nearly coplanar. Mutual inclination, $i_{mut}$, is expected to have the strongest effects on the mass transfer, thus it is a free parameter to be explored.
