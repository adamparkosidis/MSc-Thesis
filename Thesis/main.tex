%%%%%%%%%%%%%%%%%%%%%%%%%%%%%%%%%%%%%%%%%
% University of Amsterdam master thesis title page 
% LaTeX Template
%
% Version 1.2 (14/02/14) (fixed cursive mode under titlepage)
%
% This template was made for the UvA by Ludo Nieuwenhuizen
% 
% 
% Instructions for using this template:
%
% I've done my best to make this template self-explanatory. The only thing (apart from
% finishing your thesis) is that wherever you see the percentage sign after some LaTeX
% commands, you have to insert some text (except for this
% introductory part, of course)l this is also explicitly stated.
%
% This title page can be compiled as is. This is not useful for 
% including it in another document. To do this, you have two options: 
%
% 1) Copy/paste everything between \begin{document} and \end{document} 
% starting at \begin{titlepage} and paste this into another LaTeX file where you 
% want your title page.
% OR
% 2) Remove everything outside the \begin{titlepage} and \end{titlepage} and 
% move this file to the same directory as the LaTeX file you wish to add it to. 
% Then add \input{./titlepage_master_thesis.tex} to your LaTeX file where you want your
% title page.
%
% The layout is quite vulnerable for changes. If you for instance remove the logo at the
% bottom, the last line 'Department Research institute...' will get higher up the page. In this 
% case it is safer to insert a blank image or fix it in a different way.
%
% Any questions can be sent to me (L.G.Nieuwenhuizen@gmail.com)
%%%%%%%%%%%%%%%%%%%%%%%%%%%%%%%%%%%%%%%%%

\documentclass[11pt,twoside,a4paper,fleqn]{report}

\usepackage{graphicx}
\usepackage{graphics}
\usepackage[english]{babel}
\usepackage[hcentering,bindingoffset=8mm]{geometry}
\usepackage[utf8]{inputenc}
% \usepackage[en-US]{datetime2}
\usepackage{amssymb}
\usepackage{amsmath}
\usepackage{natbib}
\usepackage[colorlinks=true, allcolors=blue]{hyperref}
\usepackage{caption}
\usepackage{tocloft}
\usepackage{amsfonts}
\usepackage[section]{placeins}
\usepackage{fancyhdr}
% \usepackage{siunitx}
\usepackage{subcaption}
\usepackage{listings}
\usepackage{acro}
% \usepackage{abbriv}
\usepackage{longtable}
\usepackage{array}
\usepackage{tablefootnote}
\usepackage{epigraph}

\graphicspath{{folder/}{../figures/}}

\pagestyle{fancy}
\fancyhead[RO]{\fancyplain{}{\nouppercase{\leftmark}}}
\renewcommand\sectionmark[1]{\markboth{\MakeUppercase{#1}}{}}
\fancyhead[LO]{}
\fancyhead[LE]{\fancyplain{}{\nouppercase{\leftmark}}}
\renewcommand\sectionmark[1]{\markboth{\MakeUppercase{#1}}{}}
\fancyhead[RE]{}
\lfoot{}
\cfoot{\fancyplain{}{\thepage}}
\rfoot{}

% \sisetup{separate-uncertainty, multi-part-units = brackets}
% \DeclareSIUnit\parsec{pc}
% \DeclareSIUnit\photons{ph}
% \def\kms{km~s$^{-1}$}

\newcolumntype{L}[1]{>{\raggedright\let\newline\\\arraybackslash\hspace{0pt}}m{#1}}

%\setlength{\headheight}{15pt}

%\parskip = \baselineskip

\captionsetup[table]{name=Table,labelfont={sc,footnotesize},textfont=footnotesize,labelsep=none}
\captionsetup[figure]{name=Fig.,labelfont={sc,small},textfont=footnotesize,labelsep=endash}
\numberwithin{equation}{chapter}
\numberwithin{figure}{chapter}

\newcommand{\vdag}{(v)^\dagger}
\newcommand\aastex{AAS\TeX}
\newcommand\latex{La\TeX}

\setlength{\parindent}{0 cm}
\pagenumbering{gobble}

\acsetup{list-style=longtable}

\begin{document}
\begin{titlepage}

\newcommand{\HRule}{\rule{\linewidth}{0.8mm}}
\center
 \vspace*{0.5cm}  % Play around with this as you want
% ------
% Heading20
% ------
\raisebox{0.05cm}[0pt][0pt]{\includegraphics[width=2.0cm]{Thesis/logos/UvA_logo.png}}
\raisebox{0.7cm}[0pt][0pt]{\textsc{\Huge University of Amsterdam}}
\raisebox{-1.85cm}[0pt][0pt]{\includegraphics[width=7.0cm]{Thesis/logos/VUlogo.png}}\\[2.0 cm]

\Large{\textbf{MSc Physics and Astronomy}}\\% Study discipline
\Large{Track: Astronomy \& Astrophysics}\\[0.7cm] % Master track name
\textsc{\Large \textbf{Master Thesis}}\\[0.2cm]

% -----
% Title
% -----

\HRule \\[0.3cm]

{ \huge \bfseries The evolution of TIC 470710327 hierarchical triple system with a Roche lobe filling outer star }\\[0.8cm] % title of thesis
%{\Large \bfseries Subtitle of Thesis\\Can use two lines} % subtitle of thesis

\HRule \\[0.7cm]
 
% -----
% Details
% -----

{\Large \emph by}\\[0.6cm]
{\Large \bfseries Adam Parkosidis\\ % student name
13950142}\\[0.4cm] %student ID
%\DTMlangsetup{showdayofmonth=false}
{\large  \emph{\today}}\\ % month + year in which the thesis was concluded
%\DTMlangsetup{showdayofmonth=true}
{\large  \emph{60 ECTS}}\\ % 'this many' ects that are rewarded for the thesis
{\large  \emph{Start date - End date}}\\[1.8cm] % period in which the research was carried out

% -----
% Supervisor, etc.
% -----

\begin{minipage}{0.4\textwidth}
\begin{flushleft} \large
{\large \emph{Supervisors:}}\\

\large{Dr Silvia Toonen}\\ % Supervisor's Name
\large{Dr Silvia Toonen}
\end{flushleft}
\end{minipage}
~
\begin{minipage}{0.4\textwidth}
\begin{flushright} \large
\emph{Examiners} \\
\large{Dr Silvia Toonen} \\
\large{Oliver Porth} % Examiner's Name
\end{flushright}
\end{minipage}\\


% -----
% Logo and Department
% -----

\raisebox{-138pt}[0pt][0pt]{\includegraphics[width=5.5cm]{Thesis/logos/api_logo.pdf}}\\ % Call your institute logo "institute.jpg", or be creative.

% \raisebox{-138pt}[0pt][0pt]{\large{Anton Pannekoek Institute for Astronomy}} %name of the department or institute or company


% -----
% That was easy, right?
% -----

\vfill 

\end{titlepage}

\newpage
\chapter*{Abstract}

Mass transfer in hierarchical triple systems, and more specifically, \ac{rlof} by the outer star, is expected to be considerably different from mass transfer in an ordinary binary system. The mass cannot be simply accreted by the inner objects, but may instead form a circumbinary disk or be expelled via a slingshot effect due to the inner binary rotation. Stellar evolution, gravitational dynamics, and hydrodynamics all play important roles in the process. In the first part, we create 3D hydrodynamical models of post main-sequence stars based on detailed 1D stellar evolution models. In the second part, we use AMUSE to couple hydrodynamics with high accuracy gravitational integrators and solve these physical processes in a self-consistent  manner. Hence, we simulate the phase of mass transfer in a hierarchical triple system in which the tertiary star will overfill its Roche lobe before any of the inner stars leave the main sequence. We encounter a fairly non-conservative mass transfer, and while we quantify its impact on the inner and outer orbits, predicting the end of the mass transfer phase and the appearance of the resulting system is difficult. However, we provide some preliminary estimations of the system's accretion efficiency and the amount of angular momentum lost. Finally,
we speculate that the formation of a circumbinary disk around the inner binary probably leads to significantly more conservative mass transfer. 





\mbox{}


\newpage
\pagenumbering{roman}
\setcounter{page}{1}
{
  \hypersetup{linkcolor=black}
\tableofcontents

\newpage
\thispagestyle{empty}
\mbox{}
\newpage



\listoffigures

\newpage
\thispagestyle{empty}
\mbox{}
\newpage

\listoftables

\newpage
\thispagestyle{empty}
\mbox{}
\newpage

\mbox{}
\thispagestyle{empty}
\printacronyms[include-classes=abbrev,name=List of Abbreviations]

\newpage
\thispagestyle{empty}
\mbox{}
\newpage

\pagenumbering{arabic}
\setcounter{page}{1}

}

\parskip = 3mm

\chapter{Introduction}

\epigraph{The universe must be full of voices, calling from star to star in a myriad tongues. One day we shall join that cosmic conversation}{Arthur C. Clarke}

Observations proved that field stars are not always single; many develop in pairs, and many of these binaries are members of triples or higher-order systems. Although the fundamentals of single and binary evolution have long been acknowledged \citep{postnov2014evolution,toonen2014popcorn}, the long-term evolution of stellar triples remains unknown. In the simple case, stable triple systems, which are hierarchical, namely, consist of an inner and an outer binary orbit, i.e., the tertiary (third object). The secular evolution of such systems is the modification of orbital elements over timescales substantially larger than the system's dynamical timescale. Hence, the presence of the outer star has no influence on the history of the inner binary and the evolution of the inner binary and the tertiary can be discussed independently. In other cases, a third star in orbit around a binary system can drastically influence the system's development
via interactions, which influence the orbital elements of the inner and outer orbit through changes in energy and angular momentum. Consequently, triple stellar evolution processes which are unique to systems with multiplicities of higher orders than binaries can destabilize systems.

The rich dynamical behavior of three-body systems can produce Lidov-Kozai cycles, in which the eccentricity of the inner orbit and the inclination between the inner and outer orbits vary periodically \citep{michaely2014secular,toonen2016evolution,mangipudi2022extreme}. As a result, tidal effects (tidal friction), gravitational-wave emission, and stellar interactions such as mass transfer, angular momentum exchange and collisions may be enhanced. In this way, triples provide promising evolutionary pathways for exotic objects \citep{sana2012binary, toonen2016evolution}, e.g. blue stragglers \citep{winn2009spin}, that are difficult to be explained by binary evolution. Thus, a detailed examination of the evolution of triples demands a self consistent treatment of three-body dynamics and stellar evolution.

Additionally, the fraction of systems with companions grows with mass (see \cref{fig:stellar_companions} ), as a result massive stars are seldomly formed in isolation. In contrast, most of massive stars are created in binary or higher order multiple systems with $\sim 50\%$ of spectral type B stars be in triples \citep{sana2014southern,moe2017mind}, a percentage which reduces to $\sim 10\%$ for low-mass stars \citep{raghavan2010survey,toonen2014popcorn,moe2017mind}. In these cases, apart from the intrinsic stellar properties, the evolution depends sensitively on the interaction between the system's stellar components. Consequently, a detailed examination of stellar evolution in multiples is required for our comprehension of the life cycle of massive stars, which is still poorly understood.
\begin{figure}[H]
    \centering
    \includegraphics[width=\textwidth]{Thesis/figures/fig_moe_2017.png}
    \caption{Multiplicity fractions as a function of primary mass (dotted lines), including the single-star $F_{n=0;q> 0.1}$ (red), binary-star $F_{n=1;q> 0.1}$  (green), triple-star $F_{n=2;q> 0.1}$  (blue), and quadruple-star fraction $F_{n=3;q> 0.1}$  (magenta). Given a primary mass $M_1$, the model assumes that the multiplicity fractions follow a Poisson distribution across the interval $n = [0, 3]$ in a manner that reproduces the measured multiplicity frequency $F_{mult;q >0.1} = \Sigma_{n=1}^3 \; n F_{n;q> 0.1}$. For solar-type stars, this model matches the measured values (solid) within their uncertainties. Regardless of the uncertainties in the multiplicity fractions, $\leq 10\%$ of O-type stars are single while $\geq 55\%$ are born in triples and/or quadruples. Figure taken by \cite{moe2017mind}.}
    \label{fig:stellar_companions}
\end{figure}
In this thesis, I present the evolution of TIC 470710327 \citep{eisner2022planet}, a massive compact hierarchical triple system with a Roche lobe filling outer star. I use the Astrophysical Multipurpose Software Environment (AMUSE, \cite{portegies2018astrophysical}) to simulate the system's evolution and try to predict its future. Initially, I create stellar evolution models of the triple components until the tertiary fills its Roche lobe. I then simulate in detail the mass transfer for several orbits of the outer star using a combination of gravitational dynamics and hydrodynamics. I examine the importance of various parameters in the type of mass transfer, the effects on the inner and outer orbit, and the consequent evolution.

This thesis is structured as follows: in the remainder of this first Chapter, I will present an overview of massive star evolution until the helium burning phase with a focus on those aspects that are relevant for triple evolution. Furthermore, I will argue  how to extend these to triple evolution and introduce the the fundamentals of the Roche lobe overflow (ROLF)...


\section{Evolution of Massive Stars in Isolation}

Stars with $M \leq 2$ M$_{\odot}$, $2 < M \leq 8$ M$_{\odot}$ and $M > 8$ M$_{\odot}$ are classified as low-, intermediate-, and high-mass, respectively. Despite the inherent rarity predicted by the initial mass function (IMF, see e.g. \cite{chabrier2005initial, dib2018emergence}), massive stars play a key role in the evolution of the Universe. They are the main source of UV radiation and heavy elements. They serve as a significant source of mixing and turbulence in the interstellar medium (ISM) of galaxies through a combination of winds, outflows, expanding HII regions, and supernova explosions. Galactic dynamos are powered by turbulence in conjunction with differential rotation. Cosmic rays are accelerated by the interaction of galactic magnetic fields and supernova shock fronts. The ISM is primarily heated by cosmic rays, UV radiation, and the dissipation of turbulence, whereas it is finally cooled by heavy metals present in dust, molecules, and in atomic/ionic form. Therefore, massive stars have a significant impact on galaxies' physical, chemical, and morphological structure \citep{kennicutt2005role}. However, the physical mechanisms behind the birth, development, and demise of massive stars remain elusive in comparison with low-mass stars \citep{zinnecker2007toward}. 


\subsection{Timescales of Stellar Evolution}

The fundamental timescales of stellar evolution are the dynamical ($t_{dyn}$), thermal ($t_{th}$), and nuclear timescales ($t_{nucl}$). The dynamical timescale is the characteristic time required for a star to collapse under its own gravitational force in the absence of internal pressure:

\begin{equation}
    t_{dyn} = \sqrt{\frac{R^3}{GM}} \sim 0.02 \left( \frac{R}{R_{\odot}} \right)^{3/2} \left( \frac{M}{M_{\odot}}\right)^{1/2} \; \text{days},
\end{equation}\label{eq:dynamical_timsecale}

where $R$ and $M$ are the star's radius and mass. It is a period on which a star might expand or contract if its hydrostatic equilibrium were disrupted, e.g. in case of sudden mass loss.

Thermal (or Kelvin-Helmholtz) timescale indicates how quickly changes in a star's thermal structure may occur. It is therefore also the period on which a star responds when a its thermal equilibrium is disturbed:

\begin{equation}
    t_{th} = \frac{G M^2}{2RL} \sim 1.5 \times 10^7 \left( \frac{M}{M_{\odot}} \right)^{2} \frac{R_{\odot}}{R} \frac{L_{\odot}}{L} \; \text{yr},
\end{equation}\label{eq:thermal_timsecale}

where L is the star's luminosity.

Finally, the nuclear timescale corresponds to the time required for the star to exhaust its nuclear fuel supply at its current luminosity: 

\begin{equation}
    t_{nucl} = \frac{\phi M_{nucl} c^2}{L} \sim 10^{10} \frac{M}{M_{\odot}} \frac{L_{\odot}}{L} \; \text{yr},,
\end{equation}\label{eq:nuclear_timsecale}

where $\phi$ is the efficiency of nuclear energy production, $M_{nuc}$ is the amount of mass available as fuel, and $c$ is the light speed. For core hydrogen burning, $\phi = 0.007$ and $M_{nucl} \sim 0.1 M$.

Typically $t_{nucl} >> t_{th} >> t_{dyn}$, while assuming a mass-luminosity relation of $L \propto M^{\alpha}$, with empirically $\alpha \sim 3-4$ \citep{eker2015main}, it follows that massive stars live shorter and evolve faster than low-mass stars.

\subsection{Hertzsprung-Russell diagram}

The evolution of a star in isolation, namely single star evolution, is predominantly determined by the stellar mass. In their attempt to achieve hydrostatic and thermal equilibrium, stars generate temperatures and pressures that allow for nuclear burning. The cycles of nuclear burning and fuel exhaustion regulate the evolution of a star and set the various phases during the stellar lifetime. These burning cycles can be viewed as long-lived but transient disruptions to a star's (or at least its core's) inexorable shrinkage under the effect of gravity. The virial theorem dictates this contraction is caused by the fact that stars are hot and lose energy through radiation. 

The Hertzsprung-Russell (HR) diagram in \cref{fig:HR_massive_stars} shows four evolutionary tracks for massive stars. The black circles correspond to the Zero Age Main Sequence (ZAMS) and the Terminal Age Main Sequence (TAMS). At ZAMS, the star having started the hydrogen burning in its core, achieves thermal equilibrium (TE), $L_{nuc}/L =1$, while TAMS is defined as the core hydrogen exhaustion point. Additionally, the black squares represent the start and end of helium burning in the core, respectively. In both cases the end of nuclear burning in the core has defined as the point when hydrogen and helium core mass fractions are $< 0.01$, respectively. Finally, between TAMS and the ignition of helium, there is a short-lived phase called Hertzsprung gap branch, where hydrogen burning occurs in a shell around the core. 

\begin{figure}[H]
    \centering
    \includegraphics[width=0.9\textwidth]{Thesis/graphs/HR_massive_stars.pdf}
    \caption{Hertzsprung-Russell diagram. Evolutionary tracks for three stars in the HR-diagram with masses 9, 12 and 15 M$_{\odot}$ at solar metallicity until the end of Hellium burning. Specific moments in the evolution of the stars are noted by black circles and squares as explained in the text. The tracks are calculated with MESA (cite to be). The dashed lines show lines of constant radii by means of the Stefan–Boltzmann law.}
    \label{fig:HR_massive_stars}
\end{figure}

During the MS, hydrogen is fused into $^4$He. Independently on the ongoing reaction channel (pp or CNO), the luminosity of the star increases during this phase (L$\,\propto\,\mu^{4}\,M^3$). This is due to the change in the core's composition, as the hydrogen is fused into helium ($\mu$ increases), but the way in which the star evolves through the MS-phase depends on its mass. Stars with masses M\,$\geq$\,1.3\,M$_\odot$, consequently massive stars too, are driven by the CNO-cycle ($\epsilon_{CNO}\,\propto\,\rho\,T^{18}$). As the luminosity increases over time, the energy produced in the core needs to increase too, in order to maintain TE. However, due to the high dependence on the central temperature of the CNO-cycle, a small change of T$_c$ will have a tremendous effect on the energy production rate $\epsilon_{CNO}$, which would take out the star from HE and TE. Thus, T$_c$ needs to remain almost constant in time. The way in which such a star keeps T$_c$ constant, while $\mu$ increases, is to decrease the pressure inside the core by expanding the envelope on the nuclear time-scale, $\frac{P_c}{\rho_c} \propto \frac{T_c}{\mu}$. Hence, these stars evolve through larger radii and lower T$_{eff}$. 

Furthermore, stars driven by the CNO-cycle have convective cores, since the energy produced is too large to be transported by radiation ($\nabla_{rad} > \nabla_{ad}$). One effect of the convection on the evolution is that the MS-lifetime is extended (H is brought from the envelope inside the core, more detailed discussion in \cref{sub:mixing}). Another effect is that as the core approaches the end of H-burning, the reactions suddenly cease in the whole core. Consequently, the energy generation rate would normally decrease due to the sudden depletion of hydrogen in the core, which also diminishes the thermostatic action of the CNO reactions, threatening the thermal equilibrium of the star. In response the temperature of the core ($T_c$) needs to increase in order to keep the same energy generation rate, $\epsilon_{CNO}$. The ideal gas law implies that $\frac{P_c}{\rho_c} \propto \frac{T_c}{\mu}$, thus the pressure exerted on the core by the envelope must increase leading to the contraction of the star which also leads to higher effective temperature ($T_{eff}$). This is evident in the second part of the MS (see \cref{fig:HR_massive_stars}), where we observe the ``hook'' feature. 

At TAMS, where the hydrogen in the core has been depleted, hydrogen burning occurs in a shell around the core, while the central temperature ($T_c$) is insufficient to initiate burning in the helium core. From that point on, the star enters the Hertzsprung gap branch. The core contracts in $t_{th}$ in an attempt to reach thermal equilibrium. The virial theorem indicates that half of the gravitational energy ($E_{grav}$) is converted to internal energy ($E_{int}$) raising the central temperature ($T_c$), while the other half is escaping as luminosity ($L$) from the core. Furthermore, the hydrogen burning in the shell around the core also produces energy via the CNO-cycle. The thermostatic behavior of the CNO-cycle indicates that the burning shell must maintain thermal equilibrium by remaining at a relatively constant temperature. Because contraction of the burning shell would result in heating, the radius of the burning shell must also stay fairly constant. As the core decreases, the shell must contract, and so the pressure in the burning shell must drop. As a result, the overlaying envelope's pressure must drop, causing the layers above the shell to expand. This called the mirror principle. The aforementioned behavior is evident in Fig. \ref{fig:HR_massive_stars}, as the stars move towards bigger radius ($R$) and lower effective temperature ($T_{eff}$). An important thing to mention is that during this phase the temperature in the core ($T_c$) is raising up and should not be confused with the effective temperature ($T_{eff}$). As the star moves towards lower effective temperatures ($T_{eff}$), the opacity ($\kappa$) in the envelope raises, thus the energy transport through radiation becomes less efficient ($\nabla_{rad} \propto \kappa$), while gradually the envelope becomes convective.

\begin{comment}
A significant fraction of the energy from these two sources acts as the work applied towards the envelope leading to the expansion of the latter under hydrostatic equilibrium, while the expansion itself results to the envelope temperature being reduced. Simultaneously, the abundance of hydrogen in the shell is reducing making the shell gradually thinner, while the produced helium is adding gradually mass to the helium core speeding up its contraction.
\end{comment}

Stars of less than $12$M$_{\odot}$ reach each effective temperatures as low as ($10^{3.7}$K) 5000K before helium ignition. At this moment, they begin to ascend the red giant branch (RGB), which is accompanied by a significant rise in luminosity and radius. Due to the low effective temperature the opacity of the envelope rises and the latter becomes gradually convective. The prohibited zone of the HR-diagram is located to the right of the RGB, where hydrostatic equilibrium cannot be established. Any star in this zone will travel quickly towards the RGB. The red giant star has a compact core and an extensive envelope that extends hundreds of solar radii. When the temperature in the core exceeds $T_c \sim 10^8 K$, helium core burning begins, and the red giant phase ends. For stars with $M \geq 12$M$_{\odot}$, helium ignites before the effective temperature has dropped to a few thousand Kelvin; the movement to the right in the HR-diagram is shortened when helium ignites.

\begin{comment}
The important difference here is that there is a shell in which hydrogen burning takes places via the CNO-cycle. Consequently, the expansion of the layers around the core will result to the increase of the pressure that is exerted by the latter towards that shell. The thermostatic behavior of the CNO-cycle once again tries to maintain the thermal equilibrium, $T_{sh} \sim const.$ in the shell, and because $\frac{P_{sh}}{\rho_{sh}} \propto \frac{T_{sh}}{\mu_{sh}}$ the pressure by the envelope towards the hydrogen burning shell must increase. This leads to the envelope's contraction and consequently to the reduction of the luminosity $L$ as can be seen in Fig. \ref{fig:HR2}, which is still determined by the conditions in the photosphere because of the convective envelope. The contraction takes place until point {\it F}, but it is evident that the rate of the contraction becomes lower as the star moves towards {\it F}. In reality the temperature in the envelope slowly rises as the star reaches the end of the red-giant branch (transition from the red line to the blue line), thus the opacity also rises and the envelope becomes gradually less convective and more radiative. This is happening because not all the layers from the hydrogen burning shell and above are convective, but there is a small part at the bottom of the envelope that is still radiative. Hence, the helium burning acts now as a second source of energy inside the star. As the latter moves towards the transition point is already in thermal equilibrium and the virial theorem indicates that as the envelope contracts must rise its internal energy, thus rise its temperature. Consequently, the radiative layers above the core (not the hydrogen burning shell) are gradually heated erasing the convection from inside towards the surface of the star. From the transition point on the star enters the blue loop with a radiative envelope, which keeps contracting but the contraction is slowly decelerating, and a gradually increasing effective temperature. Hence, the luminosity starts to rise, because has a stronger dependence on the effective temperature than on the radius ($L \propto R^2 T_{eff}^4$). This is evident in Fig. \ref{fig:HR2}.

As a result, stars go through expansion and contraction phases during their lives. The change in the star's physical radius occurs at different timescales depending on the physical mechanism buried beneath these processes. For example, in nuclear timescale ($t_{nuc}$, see \eqref{eq:nuclear_timsecale}), stars expand throughout the main sequence (MS), but in thermal timescale ($t_{th}$, see \eqref{eq:thermal_timsecale}), stars expand during the H-shell burning phase.
\end{comment}



\subsection{Stellar Winds}



\subsection{Internal Mixing}\label{sub:mixing}

Even though the overall stellar evolution is only slightly affected by the initial chemical composition, a variety of internal mixing process can impact the life cycle, particularly of massive ($M>8$ M$_{\odot}$), stars \citep{langer2012presupernova}. Apart from convection, convective overshooting, semiconvection, and rotationally induced mixing are the most important internal mixing process \citep{schootemeijer2019constraining} and are still poorly understood. 

\subsubsection{Convective Overshooting}

Convective overshooting refers to the process of mixing beyond the boundaries of convective regions, which can occur when convective cells penetrate into radiative regions due to their non-zero velocity \citep{alongi1993evolutionary,brott2011rotating,schootemeijer2019constraining}. In stars with convective cores, e/g/ intermediate and massive stars, the size of the core is effectively enlarged through mechanisms such as convective core overshooting. This overshooting brings additional hydrogen into the core of the star and therefore directly impacts the final He core mass and main-sequence (MS) lifetime as well as the evolution of the stars after the MS.

\section{Binary Star Systems}

\section{Hierarchical Triple Star Systems}







\newpage

\bibliographystyle{plainnat}
\bibliography{refs}

%## IMPORTANT#############################################
\acuseall
%## IMPORTANT#############################################

\end{document}