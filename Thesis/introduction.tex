\section{Introduction}


Despite the inherent rarity predicted by the initial mass function (IMF, see e.g. \cite{chabrier2005initial, dib2018emergence}), massive stars ($M \geq 8$ M$_{\odot}$) play a key role in the evolution of the Universe. They are the main source of UV radiation and heavy elements. They serve as a significant source of mixing and turbulence in the interstellar medium (ISM) of galaxies through a combination of winds, outflows, expanding HII regions, and supernova explosions. Galactic dynamos are powered by turbulence in conjunction with differential rotation. Cosmic rays are accelerated by the interaction of galactic magnetic fields and supernova shock fronts. The ISM is primarily heated by cosmic rays, UV radiation, and the dissipation of turbulence, whereas it is finally cooled by heavy metals present in dust, molecules, and in atomic/ionic form. Therefore, massive stars have a significant impact on galaxies' physical, chemical, and morphological structure \citep{kennicutt2005role}. However, the physical mechanisms behind the birth, development, and demise of massive stars remain elusive in comparison with low-mass stars \citep{zinnecker2007toward}. 


The fraction of systems with companions grows with mass \citep{moe2017mind}, as a result massive stars are seldomly formed in isolation. In contrast, most of massive stars are created in binary or higher order multiple systems with $\sim 50\%$ of spectral type B stars be in triples \citep{sana2014southern,moe2017mind}, a percentage which reduces to $\sim 10\%$ for low-mass stars \citep{raghavan2010survey,toonen2014popcorn,moe2017mind}. Consequently, a detailed examination of stellar evolution in multiples is required for our comprehension of the life cycle of massive stars.


Although the fundamentals of single star and binary evolution have long been acknowledged \citep{postnov2014evolution,toonen2014popcorn}, the long-term evolution of stellar triples remains unknown. In the simple case, stable triple systems, which are hierarchical, namely, consist of an inner and an outer binary orbit, i.e., the tertiary (third object). The secular evolution of such systems is the modification of orbital elements over timescales substantially larger than the system's dynamical timescale. Hence, the presence of the outer star has no influence on the history of the inner binary and the evolution of the inner binary and the tertiary can be discussed independently. In other cases, a third star in orbit around a binary system can drastically influence the development the system via interactions, which influence the orbital elements of the inner and outer orbit through changes in energy and angular momentum. Consequently, triple stellar evolution processes which are unique to systems with multiplicities of higher orders than binaries can alter stable systems to become unstable.


The rich dynamical behavior of three-body systems can produce Lidov-Kozai cycles, in which the eccentricity of the inner orbit and the inclination between the inner and outer orbits vary periodically \citep{michaely2014secular,toonen2016evolution,mangipudi2022extreme}. As a result, tidal effects (tidal friction), gravitational-wave emission, and stellar interactions such as mass transfer, angular momentum exchange and collisions may be enhanced. In this way, triples provide promising evolutionary pathways for exotic objects \citep{sana2012binary, toonen2016evolution} that are difficult to be explained by binary evolution, e.g. blue straglers \citet{winn2009spin}. Thus, a detailed examination of the evolution of triples demands a self consistent treatment of three-body dynamics and stellar evolution.

\subsection{Roche Lobe Overflow}

The Roche lobe is the region around a star in a binary system within which orbiting material is gravitationally bound to that star. As a result, if a binary component fills its Roche lobe, matter might overflow to the second star, causing mass transfer. The process is called Roche lobe overflow (RLOF). The Roche lobe notion is comparable in hierarchical triple systems. The inner semi-major axis in these systems is substantially smaller than the outer, and the inner binary is practically viewed as a single mass point by the outer star. As a consequence, the mass of the tertiary and the combined mass of the inner binary constitute two Roche lobe regions.

Stars go through expansion and contraction phases during their lives. The change in the physical radius of the star occurs at different timescales depending on the physical mechanism buried beneath these processes. For example, in nuclear timescale, stars expand throughout the main sequence (MS), but in thermal timescale, stars expand during the H-shell burning phase($t_{nuc} >> t_{th}$). Subsequently, at some time during their evolution, star's physical radius may exceed its Roche lobe radius resulting to Roche lobe overflow.

In triple systems where the tertiary is the most massive star and the semi-major axis of its orbit is sufficiently small, Roche lobe overflow from the tertiary to the inner binary is anticipated at some time in its evolution \citep{de2014evolution}. In this case the outer star is projected to leave the main sequence and grow along the giant branch before any other star in the system. According to the Massive Star Catalog, this kind of mass transfer from an outer star to an inner binary should take place in $ \sim 1\%$ of all triple systems \citep{de2014evolution,hamers2022statistical}. 


\subsection{The TIC 470710327 hierarchical triple system}

A very interesting system is TIC 470710327. This massive compact hierarchical triple system, first discovered in data provided by the Transiting Exoplanet Survey Satellite (TESS) and confirmed with the HERMES spectrograph, was published by \cite{eisner2022planet}. TIC 470710327 is made up of a massive main-sequence (MS) star on an orbit of $52.04$ days and an inner circular binary of main-sequence stars with a period of $1.10$ days.  The tertiary star has a mass of $14.5–16$ M$_{\odot}$, while the individual masses of the inner binary estimated of $6-7$ M$_{\odot}$ and $5.5-6.3$ M$_{\odot}$ respectively. The tertiary's orbital arrangement is more complicated, with an eccentricity of $e = 0.3$ and a mutual inclination of $i = 16.8^{+4.2}_{-1.4} \deg$.

Such a configuration of masses is relatively rare \citep{de2014evolution} and TIC 470710327's genesis is still an open question. One would anticipate that the more massive secondary star was the first to emerge since more massive stars have shorter Kelvin-Helmholtz timescales than stars with lower masses. \cite{vigna2022mergers} presented a progenitor scenario for TIC 470710327 in which $2+2$ quadruple dynamics resulted in Zeipel–Lidov–Kozai (ZKL) oscillations triggering the merger of the more massive binary either during late phases of star formation or several $Myr$ after the zero-age main sequence (ZAMS)

Apart from the genesis of the system, also its future evolution is unknown. The measured mass ratio suggests that the $14.5–16$ M$_{\odot}$ tertiary star, which will be the first star to develop off the MS, will be the one to drive the system's further evolution. It is predicted to fill its Roche lobe and begin transferring mass to the inner binary. Although is currently dynamically stable, the outcome of such process is a strictly hydrodynamical problem \citep{de2014evolution} and depends on the nature of the mass transfer and the response of the inner binary. If the compactness of the inner binary is high enough a circumbinary disk may form. According to \cite{leigh2020mergers}, such a situation favors evolution toward equal mass inner binary stars by causing preferential accretion to the lowest mass component. On the other hand, friction may cause the inner orbit to shrink if the mass transfer stream crosses it, which might result in a contact system and/or a merger \citep{de2014evolution}. Due to the merger's rejuvenating effects as well as the accretion from the tertiary star, such a merger remnant may be regarded as a blue straggler. In the case of a merger, the triple will be reduced to a new binary system opening a new discussion about the possible fate of the system.

In this project, we select the massive compact hierarchical triple system TIC 470710327 as our research target. We use  the Astrophysical Multipurpose Software Environment (AMUSE, \cite{portegies2018astrophysical}) to simulate the evolution the system and try to predict its future. Initially, we create stellar evolution models of the triple components until the tertiary fills its Roche lobe. We then simulate in detail the  mass transfer for several orbits of the outer star using a combination of gravitational dynamics and hydrodynamics. We examine the importance of various parameters in the type of mass transfer, the consequent evolution, and the effects on the inner and outer orbit.


%According to the Massive Star Catalog, this kind of mass transfer—from an outer star to an inner binary—should take place in $ \sim 1\%$ of all triple systems \citep{de2014evolution,hamers2022statistical}. Hence, the examination of this extraordinary system promises new constraints on both the formation scenarios of multiple massive-star systems as well as their exotic evolutionary end-products.



