\section{Scientific Codes}\label{sec:scientific_codes}

In this section, I introduce the scientific codes that I coupled together in order to simulate the evolution of my target system.  Because these codes can be used in a variety of astrophysical scenarios, I will concentrate on their fundamental usage and working principles. I encourage users who want to dig deeper into the codes to follow the relevant citations. 

\subsection{MESA}

MESA (Modules for Experiments in Stellar Astrophysics, \cite{paxton2010modules,paxton2013modules,paxton2015modules,paxton2019modules}) is a powerful and versatile 1D stellar evolution code that has become one of the most widely used tools in modern astrophysics. The code is written in Fortran and is designed to solve the fully coupled structure and composition equations simultaneously, allowing for highly accurate models of stars.

MESA includes a wide range of physics modules for various astrophysical processes, such as the equation of state, nuclear reaction networks, hydrodynamics, convective and radiative energy transport, mass loss, rotation, and magnetic fields. The code can simulate the evolution of stars from their birth to their death, including complex phases such as helium and carbon burning, thermal pulses, and supernova explosions. As a result, MESA is an invaluable tool for astrophysical research, from studying the formation and evolution of stars to exploring the origins of the elements in the universe.

The central feature of MESA is the solution of the coupled structure and composition equations, which describe the internal structure of stars and the evolution of their chemical composition over time. These equations are based on fundamental principles of stellar physics, such as conservation of mass, momentum, and energy, as well as nuclear reactions that generate and consume energy in the star. The structure and composition equations can be written as a set of coupled differential equations that describe the evolution of the mass, radius, luminosity, temperature, and chemical composition of the star over time \citep{paxton2010modules,paxton2013modules,paxton2015modules,paxton2019modules}).

MESA also includes a comprehensive nuclear reaction network that describes the fusion of light elements into heavier elements in the stellar core. The network includes thousands of nuclear reactions that involve hundreds of isotopes, making it one of the most detailed and accurate nuclear reaction networks available for stellar evolution calculations.


\subsection{GADGET-2}\label{sub:gadget2}

A detailed documentation of GADGET-2 is out of the scope of this section. Nevertheless, GADGET-2, is the main code used in my simulations, thus the general description of the code will be followed by the basic principles behind the calculation of hydro- and collisionless dynamics.

GADGET-2 (GAlaxies with Dark matter and Gas intEracT) is a \ac{sph} code that simulates the gravitational and hydrodynamic evolution of collisionless and gaseous systems in astrophysical contexts \citep{springel2005cosmological}. The code is capable of modeling a wide range of physical processes, such as gas dynamics, gravity, magnetic fields, and radiative transfer. GADGET-2 is written in C++ and is publicly available under the GNU General Public License.

The hydrodynamics computation in GADGET-2 is performed by solving the equations of motion for each particle in the simulation domain. The acceleration of each particle is calculated by summing the forces acting on it, including gravity, pressure gradients, and artificial viscosity. The gravity calculation uses the hybrid TreePM method \citep{bode2000tree,bagla2002treepm}, where the simulation volume is recursively subdivided into cubic cells, with each cell containing a maximum number of particles. The algorithm then builds a tree structure where each cell is treated as a node, and nodes that are spatially close are grouped together to form larger nodes. The final tree structure is used to compute the gravitational force on each particle avoiding the need to calculate the force between all pairs of particles in the system. This method reduces the computational cost from $O(N^2)$ for direct summation to $O(N\log N)$, where $N$ is the number of particles.

The code also includes modules for modeling magnetic fields and radiative transfer. The magnetic field module includes algorithms for calculating the magnetic field evolution and its effects on the gas dynamics. The radiative transfer module includes algorithms for calculating the transport of radiation through the simulation domain and its effects on the gas and dust properties. GADGET-2 can be run in parallel on high-performance computing clusters using the Message Passing Interface (MPI) standard.

\subsubsection{Hydrodynamics}

The basic principle of the \ac{sph} method is that the fluid is represented as a set of particles with associated physical attributes such as density, pressure, and velocity. These properties are calculated at a given point in space, using the interpolation method. The method allows any function to be defined in terms of its values at a group of disordered points known as particles \citep{monaghan1982particle}. The integral interpolant of any function $f(r)$ is defined by:
\begin{equation}\label{eq:interpolant}
    \langle f(r) \rangle = \int f(r') W(r-r',h) dr'
\end{equation}
where the integration is performed across the entire space and $W$ is an interpolating/smoothing kernel. Furthermore, the method is Lagrangian, meaning that the particles move with the fluid and do not have a fixed position in space.

The smoothing kernel has two basic properties, similar with Dirac's delta function:
\begin{equation}\label{eq:kernel_property_1}
    \int W(r-r',h) dr' = 1
\end{equation}
and
\begin{equation}\label{eq:kernel_property_2}
   \lim_{h\to0} W(r-r',h) = \delta(r-r').
\end{equation}

GADGET-2 code uses the cubic spline kernel of \cite{monaghan1985refined}:
\begin{equation}\label{eq:spline_kernel}
  W(|r|,h) = \frac{1}{\pi h^3}
    \begin{cases}
      1 - \frac{3}{2}q^2 + \frac{3}{4}q^3, & 0 \leq q < 1\\
      \frac{1}{4}(2 - q)^3, & 1 \leq q <2 \\
      0, &  q \geq 2,
    \end{cases}      
\end{equation}
where $q = \frac{r}{h}$. The kernel is smooth and has a compact support, meaning it averages the properties of neighboring particles within a certain radius, called smoothing length, $h$, of the target point.  

When $q \geq 2$, there is no interaction, since $W(|r|,h)$ is zero. Thus, the amount of interactions for each particle is determined by the smoothing length $h$. When $h$ is too small, there aren't enough particles to interact with, resulting in poor precision. When $h$ is too large, local characteristics are scattered out too much, resulting in low precision and sluggish computation. Hence, the selection of the smoothing kernel, as well as an appropriate smoothing duration, is critical for both accuracy and speed. 

To make the most of \ac{sph}'s Lagrangian nature, one must accommodate for an adaptive $h$. The smoothing length should be small in high-density regions and large in low-density regions. For example, the density estimate,  which GADGET-2 does is in the form of:
\begin{equation}
    \rho_i = \sum_{j=1}^{N} m_j W(r_i -r_j,h_i),
\end{equation}
where adaptive smoothing lengths $h_i$ of each particle are designed in such a way that their kernel volumes contain a constant mass for the estimated density, implying that the smoothing lengths and estimated densities follow the (implicit) equations
\begin{equation}
    \frac{4\pi}{3} h_{i}^3 \rho_i = N_{sph} \bar{m}
\end{equation}
where $N_{sph}$ is the typical number of smoothing neighbours, and $\bar{m}$ is an average particle mass. In my simulations, I use the default value $N_{sph}=50$. 

\begin{comment}
When two gas particles approach each other the contribution of viscosity in their equation of motion cannot be ignored. Viscosity results to the transfer of momentum along velocity gradients by random motions of
the gas. To account for that, GADGET-2 (and SPH codes in general) adopts an artificial viscosity scheme.
\end{comment}


\subsubsection{Collisionless dynamics}

In self-gravitating \ac{sph} codes such as GADGET-2, the point particles represent a large amount of mass and may get arbitrarily and abnormally near in a simulation, and numerical rounding may blow up. In order to avoid particles from scattering too strongly off of one another on close approach, a softening kernel, is used to also soften gravitational forces. 

\begin{figure}[H]
    \centering
    \includegraphics[width=0.9\textwidth]{Thesis/figures/smoothening.pdf}
    \caption{The functional form of the modified potential (-), gravitational force and the density profile using the cubic spline kernel. Figure taken by \cite{price2007energy}.}
    \label{fig:smoothened_gravity}
\end{figure}

GADGET-2 employs once again the cubic spline kernel, Eq. \eqref{eq:spline_kernel}, where now $h$ refers to the softening lenght. In general the softening length can differ from the smoothing length used in the hydrodynamic calculations, but in my simulations they are equal. The basis for this is \cite{bate1997resolution} study, which demonstrated that in self-gravitating \ac{sph} simulations, employing a softening length that differs from the smoothing length might lead in unphysical results. Hence, the modified gravitational potential per unit mass is given as:
\begin{equation}\label{eq:softened_gravity}
   \Phi(r) = -G\sum_{i=1}^{N} m_i W(r-r_i,h)
\end{equation}


The functional form of the modified potential, gravitational force and the density profile using the cubic spline kernel is depicted in \cref{fig:smoothened_gravity}. It is apparent that, for $q \geq 2$, the softening is zero and the potential follows the exact form, $\Phi(r) \propto -\frac{1}{r}$. 

\subsection{Huayno}\label{sub:huayano}

Huayno \citep{pelupessy2012n} is a high-performance N-body integrator code designed to simulate the dynamics of collisionless systems, such as galaxies, star clusters, and dark matter halos. The code is written in C++ and the basic principle of Huayno is the use of a hybrid algorithm \citep{bode2000tree} that combines the particle-mesh (PM) \citep{klypin1983three} and tree-based algorithms \citep{barnes1986hierarchical,dehnen2000very} to reduce the computational cost of simulating large systems.


The PM algorithm represents the gravitational potential as a discrete mesh of fixed resolution, and the particle positions are interpolated onto the mesh using a cloud-in-cell (CIC) scheme. The gravitational forces are then calculated by solving Poisson's equation on the mesh. The tree-based algorithm, on the other hand, uses a hierarchical structure to group particles into clusters and calculates the forces between clusters at different levels of the hierarchy. By combining the advantages of both algorithms, Huayno can handle a wide range of particle distributions and non-equilibrium systems, such as systems with binary or multiple stars.




