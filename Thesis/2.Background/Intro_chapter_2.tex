\chapter{Background}\label{background}

\epigraph{The stars are not distant objects to be admired from afar. They are our partners in exploration and discovery, and we must learn to live and work with them if we are to achieve our goals.}{Arthur C. Clarke}

The evolution of a star in isolation, namely single star evolution, is predominantly determined by the stellar mass. Stars with $M \leq 2$ M$_{\odot}$, $2 < M \leq 8$ M$_{\odot}$ and $M > 8$ M$_{\odot}$ are classified as low-, intermediate-, and high-mass, respectively. In their attempt to achieve \ac{he} and \ac{te}, stars generate temperatures and pressures that allow for nuclear burning. The cycles of nuclear burning and fuel exhaustion regulate the evolution of a star and set the various phases during the stellar lifetime. These burning cycles can be viewed as long-lived, but transient disruptions to a star's (or at least its core's) inexorable shrinkage under the effect of gravity. The virial theorem dictates, that this contraction is caused by the fact that stars are hot and lose energy through radiation. 




\input{Thesis/2.Background/Section_2-1}

\section{Binary star systems}\label{sec:binary_evolution}

The evolution of a star in isolation could be summarized as:

{\it Stars contract because they are hot and lose energy through radiation (Virial Theorem), while nuclear burning cycles serve as long-lived but transitory interruptions to a star's (or at least its core's) inevitable contraction due to gravity}. 

Despite that seemingly straightforward picture, stars tend to form in multiple systems (see \cref{fig:stellar_companions}). Binaries particularly, which consist of two stars that orbit around a shared center of mass and are gravitationally connected to each other, are of immense importance. By nature, they reveal more about themselves, particularly their masses and diameters, than other astronomical objects. This is especially true for eclipsing binaries \citep{prvsa2016physics}, which provide us with direct knowledge regarding spatial relations inside the source. Furthermore, close binaries are unique cosmic laboratories providing useful insights regarding different physical process. For example, gravitational wave emission is widely studied in binary mergers where both stars are compact objects \citep{cutler1994gravitational,abbott2017gw170608,abbott2019gwtc}, accretion as a power source in X-ray binaries \citep{lewin1997x,reig2011x}, while stellar interactions such as mass transfer, angular momentum exchange and tidal friction in close binaries. 

Despite the additional complexity that these stellar interactions provide to the evolution of these systems, they offer the necessary base for discussing stellar interactions in triple systems. There are two fundamental reasons why binaries may transfer matter:
\begin{enumerate}
    \item During its evolution, one of the stars in a binary system may expand in radius (R$_{\star}$) or contract in binary separation (${\alpha}$) to the point where the companion's gravitational force can remove the outer layers of its envelope (\ac{rlof}).
    \item At some point in its evolution, usually during the late post-\ac{ms} phase, one of the stars may release most of its mass in the form of a stellar wind; part of this material will be gravitationally trapped by the companion (stellar wind accretion). 
\end{enumerate}

This section is mainly focused on interacting binaries via \ac{rlof}, (1). 

\subsection{The relative orbit}

The problem of describing the motion of two point masses
under the effect of their mutual gravity, namely the famous two-body problem, has been studied extensively and a derivation of the equations of motion is out of the project's scope. For a detailed review of the two body problem, I redirect the reader to  \cite{postnov2014evolution}. Nevertheless, based on the relative orbit model, two mutually interacting bodies' equations of motion can be simplified to a single equation representing the motion of one body in a reference frame centered on the other body. The moving body therefore behaves as if its mass was
\begin{equation}\label{eq:reduced_mass}
    \mu= \frac{M_1 M_2}{M_1 + M_2},
\end{equation}
which is known as the reduced mass. As a result, the evolution of binaries can be described by the stellar masses, $M_1$ and $M_2$, the semi-major axis, $\alpha$, and the eccentricity, $e$, of the relative orbit \citep{sana2012binary,postnov2014evolution,toonen2014popcorn}. It is worth noting that the orbital separation, is linked to the eclipse's semi-major axis, although they are not equal except in the case of a circular orbit. Additionally, $\alpha$ is commonly used in the literature to define both parameters. To avoid confusion, $r_{rel}$ refers the orbital separation of the binary components and $\alpha$ to the semi-major axis of the eclipse throughout this thesis.
\begin{figure}[H]
    \centering
    \includegraphics[width=0.9\textwidth]{Thesis/figures/relative_orbit.pdf}
    \caption{Relation between the relative orbit (left) and absolute orbits (right) of a
    binary, in this case Sirius. The black star symbol on the left figure represents the fixed star with the reference frame centered on it, while the circles represent the moving star with mass $\mu$. The absolute positions (right) are matched with the relative position of the moving body (left) via different colors.   The initial figure was taken by Frank Verbunt's lecture notes `Compact Binaries` and modified by me.}
    \label{fig:relative_orbit}
\end{figure}
The shape of the relative orbit is defined by the orbital energy and angular momentum per unit of reduced mass. For an elliptic relative orbit, specifically, the orbital energy per unit of reduced mass is
\begin{equation}\label{eq:orbital_energy}
    \epsilon = \frac{E}{\mu} = - \frac{G (M_1 + M_2)}{2\alpha}, \; \; \epsilon  < 0
\end{equation}
and the angular momentum per unit of reduced mass is
\begin{equation}\label{eq:orbital_ang_momentum}
    l = \frac{L}{\mu} = \vec{r_{rel}} \times \vec{u_{rel}} =\sqrt{G (M_1 + M_2) \alpha (1-e^2)}
\end{equation}
where $e$ is the eccentricity and $\alpha$ the semi-major axis of the eclipse, see \cref{fig:relative_orbit}. In the absence of mass loss, both $\epsilon$ and $l$ are constants of motion.  \cref{eq:orbital_energy} shows that the orbital energy is independent of the eccentricity, $e$ and in the elliptic case is always negative.

The relative position of the moving body, or the orbital separation of the binary components, see \cref{fig:relative_orbit}, is given as:
\begin{equation}\label{eq:relative_position}
    r_{rel} = \frac{\alpha (1-e^2)}{1+e \cos{\theta}}
\end{equation}
and its velocity as:
\begin{equation}\label{eq:relative_velocity}
    u_{rel}= \sqrt{GM \left( \frac{2}{r_{rel}} - \frac{1}{\alpha}\right)},
\end{equation}
where $M = M_1 + M_2$. Using \cref{eq:relative_velocity}, the semi-major axis is given as:
\begin{equation}\label{eq:semi-major_axis}
    \alpha = \frac{GM r_{rel}}{2GM - u_{rel}^2 r_{rel}}.
\end{equation}
Additionally, using  \cref{eq:orbital_ang_momentum} the eccentricity is given as:
\begin{equation}\label{eq:eccentricity}
    e = \sqrt{1 - \frac{|\vec{r_{rel}} \times \vec{u_{rel}}|^2}{G M \alpha}}.
\end{equation}

As a result, given the relative position and velocity of two stars, $M_1$ and $M_2$, the evolution of the binary can be describted using \cref{eq:orbital_energy}, \cref{eq:orbital_ang_momentum}, \cref{eq:semi-major_axis} and \cref{eq:eccentricity}.


\begin{comment}
For two stars of mass, $M_1$ at position $r_1$ and $M_2$ at position $r_2$, the the weighted mean position, namely the center of mass, can be defined as:
\begin{equation}\label{eq:center_of_mass}
    \vec{R_{cm}} = \frac{M_1 \vec{r_1} + M_2 \vec{r_2}}{M_1 + M_2}
\end{equation}
Furthermore, the reduced mass of the binary is defined as:
\begin{equation}
    \mu= \frac{M_1 M_2}{M_1 + M_2}
\end{equation}
\end{comment}


\subsection{Roche lobe overflow}\label{sub:roche_lobe}

The orbital parameters of the relative orbit are critical in defining the Roche model, a valuable tool in describing binary evolution. The Roche model describes the effective gravitational potential of the binary and it is based on three assumptions:
\begin{enumerate}
    \item The gravitational fields of both stars are assumed to be those of point masses.
    \item The binary orbit is assumed to be circular, $e=0$.
    \item The rotation of the stellar components is assumed to be synchronized with the orbital motion. 
\end{enumerate}
The Roche potential's crucial equipotential surface, which passes through the inner Lagrangian point $L_1$, defines two Roche lobes that encircle each star. Hence, each Roche lobe defines the volume in which material is gravitationally bound to the respective star. The Roche lobe can be approximated with accuracy better than $1\%$ by a sphere of radius $R_L$:
\begin{equation}\label{eq:roche_lobe}
    \frac{R_{L,1}}{\alpha} = \frac{0.49q^{2/3}}{0.6q^{2/3} + \ln{1+q^{1/3}}},
\end{equation}
where $q =M_1 / M_2$. $R_{L,2}$ is be given for $q =M_2 / M_1$. The Roche potential of a binary with $q=6.3/5.5 \approx 1.145$ and $\alpha = 1.24$ au is presented in \cref{fig:binary_equop}. The latter corresponds to the effective potential of $\xi$ Tau by replacing the inner binary masses, $M_1$ and $M_2$, with one star of mass $M = M_1 + M_2$.
\begin{figure}[H]
    \centering
    \includegraphics[width=0.9\textwidth]{Thesis/graphs/binary_equop.pdf}
    \caption{Contour plot of a binary's effective potential for $q=6.3/5.5 \approx 1.145$ and $\alpha = 1.24$ au. The five Lagrangian points are indicated as $L_1, L_2, L_3, L_4$ and $L_5$. I create the plot using the Hermite integrator which is part of AMUSE \citep{hut1995building}.}
    \label{fig:binary_equop}
\end{figure}
According to \cref{eq:roche_lobe}, the size of the Roche lobe is determined by the mass ratio $q$ and the orbital separation of the two stars, $\alpha$. The more massive star has a larger Roche lobe, whereas stars of equal mass have equal sized Roche lobes. Furthermore, the size of the Roche lobes is proportional to the orbital separation of the stars, so as the latter changes, so do the Roche lobes.

As discussed in \cref{sec:single_star_evolution}, stars contract and expand during their evolution. Additionally, the binary separation may decrease due to the loss of orbital angular momentum from the system via stellar wind mass-loss, see \cref{sub:winds} or gravitational radiation. As a result, the physical radius of a star, $R_{\star}$, may become larger than its \ac{rl}. In ths scenario, \ac{rlof,} matter from the outer layers of the Roche-lobe-filling star can freely move through the first Lagrangian point $L_1$ to the companion star. 

Mass flow via the $L_1$ point is a relatively complicated hydrodynamical problem, but the rate of mass transfer, $\dot{M}$, through $L_1$ is highly sensitive to the donor's fractional radius excess, $\frac{\Delta R}{R}$ . A comprehensive derivation of Bernoulli's law to the gas flow through the nozzle near $L_1$ results in 
\begin{equation}\label{eq:mass_loss_rate_anal}
    \dot{M} \propto \frac{M_{d}}{P} \left( \frac{\Delta R}{R}\right)^3,
\end{equation}
where $M_{d}$ is the donor mass and $P$ the orbital period of the binary.
\begin{comment}
    

Hence, the timescale of mass transfer is strongly dependent on the donor's fractional radius excess:
\begin{equation}\label{eq:mass_transfer_timescale}
   \frac{\Delta R}{R} \propto \frac{P}{t_{\dot{M}}}^{1/3}
\end{equation}
\end{comment}

According to \cref{sec:single_star_evolution}, intermediate-mass stars expand during \ac{ms} on nuclear timescale and during \ac{rgb} and \ac{agb} phases, on thermal timescale, see \cref{fig:HR_evolution}.  Hence, three cases of mass transfer can be distinguished:
\begin{enumerate}
    \item During the \ac{ms}, namely Case A
    \item During post-\ac{ms} and before helium exhaustion, namely Case B
    \item After helium exhaustion, namely Case C
\end{enumerate}
Considering that mass is the most important property for a star's evolution, see \cref{sec:single_star_evolution}, \ac{rlof} can significantly alter its evolutionary outcome.  
Thus, in a mass-transferring binary system, both stars are expected to deviate from the evolutionary paths they would have in the absence of mass transfer.

\subsection{Orbital evolution during mass transfer}\label{sub:orbit_evol_mass_loss}

During mass transfer, the transferring matter carries angular momentum causing the orbital parameters of the relative orbit to change. By differentiating \cref{eq:orbital_ang_momentum}, a general equation for the orbital evolution can be obtained:
\begin{equation}\label{eq:orb_ang_momen_derivative}
    2\frac{\dot{J}}{J} = \frac{\dot{\alpha}}{\alpha} + 2 \frac{\dot{M_1}}{M_1} + 2 \frac{\dot{M_2}}{M_2} - \frac{ \dot{M_1} + \dot{M_2}}{M_1 + M_2} - \frac{2e \dot{e}}{1-e^2},
\end{equation}
where the last term vanishes for circular orbits.

A basic assumption of \cref{eq:orb_ang_momen_derivative} is that the angular momentum stored in the rotation of the two stars is negligible compared to the orbital angular momentum. In most circumstances, this assumption is valid, however in the case of rapidly spinning objects, such as millisecond pulsars, the angular momentum contained in the pulsar's rotation may need to be considered.  

From \cref{eq:orb_ang_momen_derivative}, the evolution of the semi-major axis of the relative orbit during mass loss is given as:
\begin{equation}\label{eq:semimajor_axis_derivative}
     \frac{\dot{\alpha}}{\alpha} = 2\frac{\dot{J}}{J} - 2 \frac{\dot{M_1}}{M_1} - 2 \frac{\dot{M_2}}{M_2} + \frac{ \dot{M_1} + \dot{M_2}}{M_1 + M_2} + \frac{2e \dot{e}}{1-e^2},
\end{equation}
where the variable $\dot{J}$ represents angular momentum loss from the binary. 

The mass loss can be simply parameterized as
\begin{equation}\label{eq:mass_loss_non_cons}
    \dot{M_{a}} = - \beta \dot{M_{d}},
\end{equation}
where the subscript $a$ refers to the `accretor` and $d$ to the `donor`. In this case, $\beta$ is the fraction of the transferred mass that is accreted by the companion, and thus
\begin{equation}\label{eq:mass_loss_non_cons_2}
    \dot{M_{a}} + \dot{M_{d}} = (1-\beta) \dot{M_{d}}.
\end{equation}

In the ideal situation where all the transferred mass from the first star is accreted by the second, $\beta=1$, the orbital angular momentum of the binary is conserved, $\dot{J}=0$ and \cref{eq:semimajor_axis_derivative} reduces to
\begin{equation}\label{eq:orb_ang_momen_derivative_cons}
    \frac{\dot{\alpha}}{\alpha}= 2 \left( \frac{M_d}{M_a} - 1 \right) \frac{\dot{M_{d}}}{M_{d}} + \frac{2e \dot{e}}{1-e^2}.
\end{equation}    
This the case of conservative mass transfer and a close look to \cref{eq:orb_ang_momen_derivative_cons} reveals that for a circular orbit the orbital separation shrinks as long as $M_d > M_a$, while it expands when $M_d < M_a$. Finally, the minimum orbital distance is given for $M_d = M_a$. 

In the general case of \ac{rlof}, mass transfer is expected to be non-conservative meaning that  $\beta < 1$. Furthermore,
the ejected mass carries away orbital angular momentum from the system. The amount of specific angular momentum that is lost can be parameterized to be $\eta$ times the specific angular momentum of the binary:
\begin{equation}\label{eq:orb_ang_mom_non_cons}
    \frac{\dot{J}}{J} = \eta \frac{\dot{M_{a}} + \dot{M_{d}}}{M_{a} + M_{d}}.
\end{equation}
Using \cref{eq:mass_loss_non_cons_2} and \cref{eq:orb_ang_mom_non_cons}, \cref{eq:semimajor_axis_derivative} gives:
\begin{equation}\label{eq:orb_ang_momen_derivative_non_cons}
    \frac{\dot{\alpha}}{\alpha}= -2\frac{\dot{M_{d}}}{M_{d}} \left( 1- \beta\frac{M_d}{M_a} - (1-\beta)(\eta + \frac{1}{2}) \frac{M_d}{M_d + M_a}  \right)  + \frac{2e \dot{e}}{1-e^2}.
\end{equation}  
The parameters $\beta$ and $\eta$ are highly uncertain. Depending on the mass transfer mode, $\eta$ can be specified in terms of other quantities \citep{postnov2014evolution}.

Assuming circular orbit, $\beta$ and $\eta$ to be constants and by ignoring the internal angular momentum of the stars, \cite{portegies1995formation} shows that the evolution of the semi-major axis based on mass redistribution can be described as:
\begin{equation}\label{eq:semimajor_axis_no_cons}
    \frac{\alpha}{\alpha_{0}} = \left (\frac{M_{d} M_{a}}{M_{d,0} M_{a,0}} \right)^{-2} \left (\frac{M_{d} + M_{a}}{M_{d,0}+M_{a,0}} \right)^{2\eta +1},
\end{equation}    
where the `0` subscript corresponds to the initial values, before \ac{rlof}.
    






\section{Hierarchical triple star systems}\label{sec:triples_evolution}

The majority triple star systems are observed in hierarchical structures. More specifically, they consist of an inner binary and a distant star (hereafter tertiary/outer star) that orbits the center of mass of the inner binary system such as $\frac{\alpha_{in}}{\alpha_{out}} << 1$. A schematic of a hierarchical triple system is presented in \cref{fig:triple_schem}.
\begin{figure}[H]
    \centering
    \includegraphics[width=0.9\textwidth]{Thesis/figures/triple_schem.pdf}
    \caption{A schematic of a hierarchical triple system consisting of an inner and an outer binary. The inner binary consists of objects whose masses are $m_1$ and $m_2$, and the outer one is the pair of the inner binary and the third body with mass $m_3$. Figure taken by \cite{gupta2020gravitational}.}
    \label{fig:triple_schem}
\end{figure}
In certain situations, the presence of the outer star does not alter the evolution of the inner binary. In such cases, the evolution of the tertiary  and the inner binary can be studied separately. In other circumstances, the three stars interact in ways that are unique to systems with multiplicities higher than binaries. As a result, numerous new evolutionary paths are anticipated compared to binary evolution. In the following subections, I focus on the dynamical stability of triple systems and the Lidov-Kozai cycles. These topics are relevant to my work, but for a detailed overview of triple evolution I direct the reader to \cite{michaely2014secular,toonen2016evolution}.

\subsection{Stability of triples}\label{sub:stability_triples}

Unlike two-body systems, the three-body problem does not have closed-form solutions. On the one hand, triple systems in the unstable regime tend to disintegrate into lower order systems on dynamical timescales \citep{van2007formation}. On the other hand, stability can occur (and last) on different timescales, thus it is not trivial to draw a clear line between stable and unstable triple systems. 

\cite{mardling1999dynamics} stability criterion is usually used in studies of triple systems' evolution, where a system is unstable if
$\frac{\alpha_{in}}{\alpha_{out}} < \frac{\alpha_{in}}{\alpha_{out}} |_{crit}$. The critical fraction is given as:
\begin{equation}\label{eq:stability_regime}
    \frac{\alpha_{in}}{\alpha_{out}} |_{crit} = \frac{2.8}{1-e_{out}} (1- \frac{0.3 i_{mut}}{\pi}) \left ( \frac{(1 + q_{out})(1+e_{out})}{\sqrt{1-e_{out}}} \right )^{2/5},
\end{equation}
where $q_{out} \equiv m_3 / (m_1 + m_2)$. The criterion is based on the concept of chaos and the result of overlapping resonances. The criterion is conservative since the existence of chaos is not always synonymous with instability. Many more stability criteria have been proposed and I direct the reader to \cite{mardling2001stability,georgakarakos2008stability}.

In \cref{sub:orbit_evol_mass_loss} I discussed how the orbital parameters change during non conservative mass transfer. These concepts are also applicable for triples, where the orbital parameters of the inner and the outer orbit can change due to angular momentum loss from the system. As a result, initially stable triple systems can become dynamically unstable during their evolution.

\subsection{Lidov-Kozai cycles}\label{sub:lidov_kozai}

Dynamical interactions in hierarchical triples can differ from the case of binaries. The presence of the third star can give rise to the Lidov-Kozai mechanism \citep{lidov1962evolution,kozai1962secular}, which can have a significant impact on the secular evolution the system. During Lidov-Kokai cycles, the mutual torque between the inner and outer binary orbits result to angular momentum exchange. Furthermore, the orbital energy is preserved, and hence the semi-major axes are also conserved. Consequently, the orbital inner eccentricity and mutual inclination vary periodically (i.e. 'cycles'). When the inclination between the two orbits is minimized, the inner binary's eccentricity reaches its maximum. Furthermore, the pericenter argument may rotate or librate periodically.

Evolution during Lidov-Kozai cycles can be fairly complicated, but given some assumptions, analytical expressions can be derived. For example, in the test-particle approximation ($e_{in}=e_{out}=0$ and $M_2 << M_1, M_3$ \citep{naoz2013secular}), the mechanism is expected to take place only if $i_{mut} \in [39.2^{\circ},140.8^{\circ}$]. Furthermore, by expanding the three-body Hamiltonian to quadrupole order in $a_{in}/a_{out}$ \citep{kinoshita1999analytical}, the timescale for the Lidov-Kozai cycles is
\begin{equation}\label{eq:lidov_kozai_timescale}
    t_{kozai} \approx \frac{P_{out}^2}{P_{in}} + \frac{M_1 + M_2 +M_3}{M_3}(1-e_{out}^2)^{3/2},
\end{equation}
where $P_{in}$ and $P_{out}$ are the periods of the inner and outer orbit, respectively. Typically, $t_{kozai} >> P_{in},P_{out}$.

Higher orders of $a_{in}/a_{out}$, i.e. the octupole level of approximation, result to even richer dynamical behavior than the quadrupole approximation. The octupole term is expected to be important when $\epsilon_{oct} \geq 0.01$ \citep{naoz2011hot,shappee2013mass} and
\begin{equation}\label{eq:octupole_term}
    \epsilon_{oct} = \frac{M_1 - M_2}{M_1 + M_2} \frac{a_{in}}{a_{out}} \frac{e_{out}}{1-e_{out}^2}.
\end{equation}

\section{Scientific Codes}\label{sec:scientific_codes}

In this section, I introduce the scientific codes that I coupled together in order to simulate the evolution of my target system.  Because these codes can be used in a variety of astrophysical scenarios, I will concentrate on their fundamental usage and working principles. I encourage users who want to dig deeper into the codes to follow the relevant citations. 

\subsection{MESA}

MESA (Modules for Experiments in Stellar Astrophysics, \cite{paxton2010modules,paxton2013modules,paxton2015modules,paxton2019modules}) is a powerful and versatile 1D stellar evolution code that has become one of the most widely used tools in modern astrophysics. The code is written in Fortran and is designed to solve the fully coupled structure and composition equations simultaneously, allowing for highly accurate models of stars.

MESA includes a wide range of physics modules for various astrophysical processes, such as the equation of state, nuclear reaction networks, hydrodynamics, convective and radiative energy transport, mass loss, rotation, and magnetic fields. The code can simulate the evolution of stars from their birth to their death, including complex phases such as helium and carbon burning, thermal pulses, and supernova explosions. As a result, MESA is an invaluable tool for astrophysical research, from studying the formation and evolution of stars to exploring the origins of the elements in the universe.

The central feature of MESA is the solution of the coupled structure and composition equations, which describe the internal structure of stars and the evolution of their chemical composition over time. These equations are based on fundamental principles of stellar physics, such as conservation of mass, momentum, and energy, as well as nuclear reactions that generate and consume energy in the star. The structure and composition equations can be written as a set of coupled differential equations that describe the evolution of the mass, radius, luminosity, temperature, and chemical composition of the star over time \citep{paxton2010modules,paxton2013modules,paxton2015modules,paxton2019modules}).

MESA also includes a comprehensive nuclear reaction network that describes the fusion of light elements into heavier elements in the stellar core. The network includes thousands of nuclear reactions that involve hundreds of isotopes, making it one of the most detailed and accurate nuclear reaction networks available for stellar evolution calculations.


\subsection{GADGET-2}\label{sub:gadget2}

A detailed documentation of GADGET-2 is out of the scope of this section. Nevertheless, GADGET-2, is the main code used in my simulations, thus the general description of the code will be followed by the basic principles behind the calculation of hydro- and collisionless dynamics.

GADGET-2 (GAlaxies with Dark matter and Gas intEracT) is a smoothed particle hydrodynamics (SPH) code that simulates the gravitational and hydrodynamic evolution of collisionless and gaseous systems in astrophysical contexts \citep{springel2005cosmological}. The code is capable of modeling a wide range of physical processes, such as gas dynamics, gravity, magnetic fields, and radiative transfer. GADGET-2 is written in C++ and is publicly available under the GNU General Public License.

The hydrodynamics computation in GADGET-2 is performed by solving the equations of motion for each particle in the simulation domain. The acceleration of each particle is calculated by summing the forces acting on it, including gravity, pressure gradients, and artificial viscosity. The gravity calculation uses the hybrid TreePM method \citep{bode2000tree,bagla2002treepm}, where the simulation volume is recursively subdivided into cubic cells, with each cell containing a maximum number of particles. The algorithm then builds a tree structure where each cell is treated as a node, and nodes that are spatially close are grouped together to form larger nodes. The final tree structure is used to compute the gravitational force on each particle avoiding the need to calculate the force between all pairs of particles in the system. This method reduces the computational cost from $O(N^2)$ for direct summation to $O(N\log N)$, where $N$ is the number of particles.

The code also includes modules for modeling magnetic fields and radiative transfer. The magnetic field module includes algorithms for calculating the magnetic field evolution and its effects on the gas dynamics. The radiative transfer module includes algorithms for calculating the transport of radiation through the simulation domain and its effects on the gas and dust properties. GADGET-2 can be run in parallel on high-performance computing clusters using the Message Passing Interface (MPI) standard.

\subsubsection{Hydrodynamics}

The basic principle of the SPH method is that the fluid is represented as a set of particles with associated physical attributes such as density, pressure, and velocity. These properties are calculated at a given point in space, using the interpolation method. The method allows any function to be defined in terms of its values at a group of disordered points known as particles \citep{monaghan1982particle}. The integral interpolant of any function $f(r)$ is defined by:
\begin{equation}\label{eq:interpolant}
    \langle f(r) \rangle = \int f(r') W(r-r',h) dr'
\end{equation}
where the integration is performed across the entire space and $W$ is an interpolating/smoothing kernel. Furthermore, the method is Lagrangian, meaning that the particles move with the fluid and do not have a fixed position in space.

The smoothing kernel has two basic properties, similar with Dirac's delta function:
\begin{equation}\label{eq:kernel_property_1}
    \int W(r-r',h) dr' = 1
\end{equation}
and
\begin{equation}\label{eq:kernel_property_2}
   \lim_{h\to0} W(r-r',h) = \delta(r-r')
\end{equation}

GADGET-2 code uses the cubic spline kernel of \cite{monaghan1985refined}:
\begin{equation}\label{eq:spline_kernel}
  W(|r|,h) = \frac{1}{\pi h^3}
    \begin{cases}
      1 - \frac{3}{2}q^2 + \frac{3}{4}q^3, & 0 \leq q < 1\\
      \frac{1}{4}(2 - q)^3, & 1 \leq q <2 \\
      0, &  q \geq 2
    \end{cases}       
\end{equation}
where $q = \frac{r}{h}$. The kernel is smooth and has a compact support, meaning it averages the properties of neighboring particles within a certain radius, called smoothing length, $h$, of the target point.  

When $q \geq 2$, there is no interaction, since $W(|r|,h)$ is zero. Thus, the amount of interactions for each particle is determined by the smoothing length $h$. When $h$ is too small, there aren't enough particles to interact with, resulting in poor precision. When $h$ is too large, local characteristics are scattered out too much, resulting in low precision and sluggish computation. Hence, the selection of the smoothing kernel, as well as an appropriate smoothing duration, is critical for both accuracy and speed. 

To make the most of SPH's Lagrangian nature, one must accommodate for an adaptive $h$. The smoothing length should be small in high-density regions and large in low-density regions. For example, the density estimate,  which GADGET-2 does is in the form of:
\begin{equation}
    \rho_i = \sum_{j=1}^{N} m_j W(r_i -r_j,h_i)
\end{equation}
where adaptive smoothing lengths $h_i$ of each particle are designed in such a way that their kernel volumes contain a constant mass for the estimated density, implying that the smoothing lengths and estimated densities follow the (implicit) equations
\begin{equation}
    \frac{4\pi}{3} h_{i}^3 \rho_i = N_{sph} \bar{m}
\end{equation}
where $N_{sph}$ is the typical number of smoothing neighbours, and $\bar{m}$ is an average particle mass. In my simulations, I use the default value $N_{sph}=50$. 

\begin{comment}
When two gas particles approach each other the contribution of viscosity in their equation of motion cannot be ignored. Viscosity results to the transfer of momentum along velocity gradients by random motions of
the gas. To account for that, GADGET2 (and SPH codes in general) adopts an artificial viscosity scheme.
\end{comment}


\subsubsection{Collisionless Dynamics}

In self-gravitating SPH codes such as GADGET-2, the point particles represent a large amount of mass and may get arbitrarily and abnormally near in a simulation, and numerical rounding may blow up. In order to avoid particles from scattering too strongly off of one another on close approach, a softening kernel, is used to also soften gravitational forces. 

GADGET-2 employs once again the cubic spline kernel, Eq. \eqref{eq:spline_kernel}, where now $h$ refers to the softening lenght. In general the softening length can differ from the smoothing length used in the hydrodynamic calculations, but in my simulations they are equal. The basis for this is \cite{bate1997resolution} study, which demonstrated that in self-gravitating SPH simulations, employing a softening length that differs from the smoothing length might lead in unphysical results. Hence, the modified gravitational potential per unit mass is given as:
\begin{equation}\label{eq:softened_gravity}
   \Phi(r) = -G\sum_{i=1}^{N} m_i W(r-r_i,h)
\end{equation}


The functional form of the modified potential, gravitational force and the density profile using the cubic spline kernel is depicted in \cref{fig:smoothened_gravity}.
\begin{figure}[H]
    \centering
    \includegraphics[width=0.9\textwidth]{Thesis/figures/smoothening.pdf}
    \caption{The functional form of the modified potential (-), gravitational force and the density profile using the cubic spline kernel. Figure taken by \cite{price2007energy}.}
    \label{fig:smoothened_gravity}
\end{figure}
It is apparent that, for $q \geq 2$, the softening is zero and the potential follows the exact form, $\Phi(r) \propto -\frac{1}{r}$. 


\subsection{Huayno}\label{sub:huayano}

Huayno \citep{pelupessy2012n} is a high-performance N-body integrator code designed to simulate the dynamics of collisionless systems, such as galaxies, star clusters, and dark matter halos. The code is written in C++ and the basic principle of Huayno is the use of a hybrid algorithm \citep{bode2000tree} that combines the particle-mesh (PM) \citep{klypin1983three} and tree-based algorithms \citep{barnes1986hierarchical,dehnen2000very} to reduce the computational cost of simulating large systems.


The PM algorithm represents the gravitational potential as a discrete mesh of fixed resolution, and the particle positions are interpolated onto the mesh using a cloud-in-cell (CIC) scheme. The gravitational forces are then calculated by solving Poisson's equation on the mesh. The tree-based algorithm, on the other hand, uses a hierarchical structure to group particles into clusters and calculates the forces between clusters at different levels of the hierarchy. By combining the advantages of both algorithms, Huayno can handle a wide range of particle distributions and non-equilibrium systems, such as systems with binary or multiple stars.




