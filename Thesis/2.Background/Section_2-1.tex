


\begin{comment}
 Despite the inherent rarity predicted by the initial mass function (IMF, see e.g. \cite{chabrier2005initial, dib2018emergence}), massive stars play a key role in the evolution of the Universe. They are the main source of UV radiation and heavy elements. They serve as a significant source of mixing and turbulence in the interstellar medium (ISM) of galaxies through a combination of winds, outflows, expanding HII regions, and supernova explosions. Galactic dynamos are powered by turbulence in conjunction with differential rotation. Cosmic rays are accelerated by the interaction of galactic magnetic fields and supernova shock fronts. The ISM is primarily heated by cosmic rays, UV radiation, and the dissipation of turbulence, whereas it is finally cooled by heavy metals present in dust, molecules, and in atomic/ionic form. Therefore, massive stars have a significant impact on galaxies' physical, chemical, and morphological structure \citep{kennicutt2005role}. However, the physical mechanisms behind the birth, development, and demise of massive stars remain elusive in comparison with low-mass stars \citep{zinnecker2007toward}. 
\end{comment}

\section{Timescales of stellar evolution}\label{sec:timescales}

The fundamental timescales of stellar evolution are the dynamical, thermal and nuclear timescales. The dynamical timescale is the characteristic time required for a star to collapse under its own gravitational force in the absence of internal pressure:
\begin{equation}\label{eq:dynamical_timsecale}
    t_{dyn} = \sqrt{\frac{R^3}{GM}} \sim 0.02 \left( \frac{R}{R_{\odot}} \right)^{3/2} \left( \frac{M}{M_{\odot}}\right)^{1/2} \; \text{days},
\end{equation}
where $R$ and $M$ are the star's radius and mass. It is a period on which a star might expand or contract if its hydrostatic equilibrium were disrupted, e.g. in case of sudden mass-loss.

Thermal (or Kelvin-Helmholtz) timescale indicates how quickly changes in a star's thermal structure may occur. It is therefore also the period on which a star responds when its thermal equilibrium is disturbed:
\begin{equation}\label{eq:thermal_timsecale}
    t_{th} = \frac{G M^2}{2RL} \sim 1.5 \times 10^7 \left( \frac{M}{M_{\odot}} \right)^{2} \frac{R_{\odot}}{R} \frac{L_{\odot}}{L} \; \text{yr},
\end{equation}
where L is the star's luminosity.

Finally, the nuclear timescale corresponds to the time required for the star to exhaust its nuclear fuel supply at its current luminosity: 
\begin{equation}\label{eq:nuclear_timsecale}
    t_{nuc} = \frac{\phi M_{nucl} c^2}{L} \sim 10^{10} \frac{M}{M_{\odot}} \frac{L_{\odot}}{L} \; \text{yr},
\end{equation}
where $\phi$ is the efficiency of nuclear energy production, $M_{nuc}$ is the amount of mass available as fuel, and $c$ is the light speed. For core hydrogen burning, $\phi = 0.007$ and $M_{nucl} \sim 0.1 M$ \citep{pols2011stellar}.

Typically $t_{nuc} >> t_{th} >> t_{dyn}$, while assuming a mass-luminosity relation of $L \propto M^{\alpha}$, with empirically $\alpha \sim 3-4$ \citep{eker2015main}, it follows that intermediate- and high-mass stars live shorter and evolve faster than low-mass stars.

%\subsection{Hertzsprung-Russell Diagram}\label{sub:HR_diagram}

\section{Evolution of intermediate-mass stars in isolation}\label{sec:single_star_evolution}

\begin{comment}

The late stages of their evolution, namely the Asymptotic Giant Branch (AGB) phase, is characterized by rich nucleosynthesis and strong mass loss, that eventually removes their envelopes, leaving behind a degenerate C-O core as the central star of a planetary nebula\citep{pols2011stellar}. Additionally, the lost mass provides material for future generations of stars, 
\end{comment}


The \ac{hrd} diagram is a useful tool for studying stellar evolution. It is a logarithmic representation of the surface luminosity (or absolute magnitude) of a star as a function of its effective temperature (or spectral type). Furthermore, it is divided into five different regions, namely the \ac{ms}, the \ac{rgb}, the horizontal-giant branch, the \ac{agb} and the region of degenerate stars. A star, depending on its mass, will pass through some of these regions during its lifetime as a result of the interplay between nuclear reactions and gravitational contraction. 

\begin{figure}[H]
    \centering
    \includegraphics[width=0.9\textwidth]{Thesis/graphs/HR_inter_stars.pdf}
    \caption{Hertzsprung-Russell diagram. Evolutionary tracks for three stars with masses 3, 5.5 and 8 M$_{\odot}$ at solar metallicity until the end of Helium burning. Specific moments in the evolution of the stars are noted by black circles and squares as explained in the text. I calculate the tracks using MESA \citep{paxton2010modules,paxton2013modules,paxton2015modules,paxton2019modules}. The dashed lines show lines of constant radii by means of the Stefan–Boltzmann law.}
    \label{fig:HR_inter_stars}
\end{figure}
Intermediate-mass stars have a complex evolutionary path that includes multiple phases of fusion and contraction. The \ac{hrd} diagram in \cref{fig:HR_inter_stars} shows three evolutionary tracks for intermediate-mass stars. From left to right, the black circles correspond to the \ac{zams} and \ac{tams}. At \ac{zams}, the star having started the hydrogen burning in its core, achieves \ac{te}, $L_{nuc}/L =1$, while \ac{tams} is defined as the core hydrogen exhaustion point. Additionally, the black squares represent the start and end of helium burning in the core, respectively.
\ac{tams} and the end of helium burning in the core are both defined as the point at which the hydrogen and helium core mass fractions are $< 0.01$, respectively. The change in the star's physical radius occurs on different timescales depending on the physical mechanism buried beneath these processes. On nuclear timescale, see \cref{eq:nuclear_timsecale}), stars evolve throughout the \ac{ms} and the helium-burning phase, while on thermal timescale (see  \cref{eq:thermal_timsecale}), stars expand during the H-shell burning phase.
\begin{figure}[H]
    \centering
    \includegraphics[width=0.9\textwidth]{Thesis/graphs/HR_evolution.pdf}
    \caption{Evolution of 5.5 M$_{\odot}$ at solar metallicity until a the end of the early \ac{agb} phase. The evolutionary phases are categorized in different colors as explained in the text. I calculate the tracks using MESA \citep{paxton2010modules,paxton2013modules,paxton2015modules,paxton2019modules}. The dashed lines show lines of constant radii by means of the Stefan–Boltzmann law.}
    \label{fig:HR_evolution}
\end{figure}
\cref{fig:HR_evolution} illustrates the evolution of a 5.5 M$_{\odot}$ star in isolation from the beginning of the \ac{ms} until the end of the early \ac{agb} phase. The evolutionary track is further categorized into sections and subsections: 
\begin{enumerate}
    \item Main-Sequence (green)
    \item Giant branch: 
        \subitem Hertzsprung gap or subgiant branch (yellow) 
        \subitem Red giant branch (red)
    \item Blue loop (blue)
    \item Asymptotic giant branch (black)
\end{enumerate}
During \ac{ms}, hydrogen is fused into $^4$He. Independent of the ongoing reaction channel (pp or CNO), the luminosity of the star increases during this phase, (L$\,\propto\,\mu^{4}\,M^3$), due to the change in the core's composition ($\mu$ increases). Nevertheless, the way in which a star evolves through the \ac{ms}-phase depends on its mass. Intermediate-mass stars have masses M\,$\geq$\,1.3\,M$_\odot$ and are driven by the CNO-cycle ($\epsilon_{CNO}\,\propto\,\rho_{c}T^{18}_{c}$). The thermostatic action of the CNO-cycle causes the envelope to expand and, as a result, a decrease in effective temperature. In general, stars driven by the CNO-cycle evolve through larger radii and lower T$_{eff}$ than stars driven by the pp-cycle (low-mass stars). Additionally, intermediate-mass stars have convective cores, since the energy produced is too large to be transported by radiation ($\nabla_{rad} > \nabla_{ad}$). One effect of the convection on the evolution is that the \ac{ms}-lifetime is extended (more detailed discussion in \cref{sec:convection}). Another effect is that as the core approaches the end of H-burning, the reactions suddenly cease in the whole core, threatening the thermal equilibrium of the star. In response the temperature of the core needs to increase in order to keep the same energy generation rate leading to the contraction of the star. This is evident in the second part of the \ac{ms} (see \cref{fig:HR_evolution}), where we observe the ``hook'' feature. 

At \ac{tams}, when the hydrogen in the core has been depleted, hydrogen burning occurs in a shell around it, while the central temperature is insufficient to initiate He-burning. From that point on, the evolution is driven by the mirror principle and stars evolve towards red giants \citep{pols2011stellar}. The core contracts on $t_{th}$ in an attempt to reach \ac{te}, while the envelope expands. The aforementioned behavior is evident in \cref{fig:HR_inter_stars} and \cref{fig:HR_evolution}, as stars move towards bigger radii and lower effective temperatures. Intermediate-mass stars reach effective temperatures as low as 5000K ($\sim 10^{3.7}$K) before helium ignition. At this moment, they begin to ascend the \ac{rgb}, which is accompanied by a significant rise in luminosity and radius. Due to the low effective temperature the opacity of the envelope rises and the latter becomes gradually convective ($\nabla_{rad} \propto \frac{\kappa}{T^4}$, more detailed discussion in \cref{sec:convection}). The prohibited zone of the HR-diagram is located to the right of the \ac{rgb}, where hydrostatic equilibrium cannot be established. Any star in this zone will travel quickly towards the \ac{rgb}. The red giant star has a compact core and an extensive envelope that extends hundreds of solar radii. 

When the temperature in the core exceeds $T_c \sim 10^8 K$, helium core burning begins. $^4$He fuses into a mixture of $^{12}$C and $^{16
}$O via the $triple- \alpha$ and $C+\alpha$ reactions, respectively \citep{pols2011stellar}. Helium ignition is thermally stable for intermediate-mass stars. The contraction of the core seizes, the envelope starts to shrink and the stellar radius decreases on $t_{nuc}$. As the star begins to descend the \ac{rgb}, the temperature in the envelope gradually rises. As a result, the envelope progressively shifts from convective to radiative. From the transition point on, the star enters the blue loop with a helium burning core, a hydrogen burning shell, a radiative envelope, and a gradually increasing effective temperature, see \cref{fig:HR_evolution}.  Hence, the luminosity starts to rise, because it has a stronger dependence on the effective temperature than on the radius ($L \propto R^2 T_{eff}^4$). After the point where star reaches local maximum temperature, see \cref{fig:HR_evolution}, the evolution is similar with the second part of the \ac{ms} (``hook'' feature). Due to the convective core, the He-burning reactions suddenly cease in the whole core and not gradually. The temperature of the core ($T_c$) needs to increase in order to keep the same energy generation rate thus the pressure exerted on the core by the layers between the hydrogen burning shell and the core must also increase. This result to the decrease of the pressure that is exerted by these layers towards the hydrogen burning shell. Due to the thermostatic behavior of the CNO-cycle the pressure by the envelope towards the hydrogen burning shell must also decrease. This leads to the envelope's expansion and the reduction of the effective temperature until the end of He-burning, see Fig. \ref{fig:HR_evolution}.

The end of He-burning marks the beginning of the \ac{agb} phase (black part of the track in Fig. \ref{fig:HR_evolution}). This is a brief yet very important evolutionary phase of intermediate-mass stars. It is characterized by rich nucleosynthesis and the enrichment of the outer layers with heavy elements. More specifically, for stars with $M \geq 4M_{\odot}$, the convective envelope can penetrate down into the helium rich layers, dredging up He- and N-rich material. The late \ac{agb} phase is characterized by strong mass loss, that eventually removes the stellar envelopes, leaving behind a degenerate C-O core as the central star of a planetary nebula\citep{pols2011stellar}. Hence understanding the nature and history of intermediate-mass stars is critical for investigating chemical enrichment and energy generation in galaxies, as well as forecasting the fate of stars like our Sun. For a more in depth description of the \ac{ms} and post-\ac{ms} evolution, I direct the reader to \cite{pols2011stellar}.

\subsection{Stellar winds}\label{sub:winds}

Stellar winds play an important role in stellar evolution because they cause mass and angular momentum loss from stars. Even though they are fundamental in understanding the formation and evolution of stars, the stellar wind mechanisms involved are not well understood in many cases. Hence, $\dot{M}$ is frequently quite uncertain introducing significant complexities in the evolution of stars. Driven by observations and theoretical models, stellar winds are usually categorized in hot and cold winds, because they are generated by different mechanisms. Numerous mass-loss rate prescriptions have been proposed, with $\dot{M}$ varying in different parts of the HR diagram.

Hot, luminous stars (mostly OB-type \ac{ms} stars and blue supergiants) experience a stellar wind driven by radiation. Radiation pressure causes an outward acceleration at frequencies corresponding to absorption lines in the spectrum, where the interaction between photons and matter is strong. Because it is primarily the lines of the heavier elements that contribute to line driving, radiation-driven mass-loss is dependent also on metallicity \citep{vink2001mass}. Mass loss due to hot winds is relatively unimportant for intermediate-mass stars.

Cool, luminous \ac{rsg} produce a stellar wind, which is likely caused by a combination of stellar pulsations and radiation pressure on dust particles that accumulate in the cool outer atmosphere. Because there are no theoretical predictions, empirical formulas that fit the average observed mass-loss rates of stars of roughly solar metallicity are used in theoretical studies. For example, Reimers' mass-loss empirical formula \citep{reimers1975circumstellar} is commonly used to model the mass-loss on the \ac{rgb}:
\begin{equation}\label{eq:reimer}
    \dot{M}_{R} = -4 \times 10^{-13} \eta 
    \frac{L}{L_{\odot}}  \frac{R}{R_{\odot}} \frac{M_{\odot}}{M},
    \text{ M${\odot} \; yr^{-1}$},
\end{equation}
where $\eta$ is parameter of the order of unity.

During the final stages of evolution on the \ac{agb}, low- and intermediate-mass stars' outer atmospheres are cool enough that dust formation becomes significant enough to dominate the opacity. Such stars experience significant mass-loss, which removes the remaining envelope, leaving only a C-O \ac{wd} as their final remnant \citep{marigo2007evolution, pols2011stellar}. Blocker's mass-loss empirical formula \citep{bloecker1995stellarI,bloecker1995stellarII} is commonly used to model the mass-loss on the \ac{agb}:
\begin{equation}\label{eq:blocker}
    \dot{M}_{Bl} = -4.83 \times 10^{-9} M^{-2.1} L^{2.7} \dot{M}_{R},
    \text{ M${\odot} \; yr^{-1}$}.
\end{equation}

Other mass-loss prescriptions that are common in theoretical studies are given by \cite{de1988mass,nieuwenhuijzen1990parametrization}. The high mass-loss rate during the \ac{agb} phase defines both the maximum luminosity that a low- or intermediate-mass star may achieve on the \ac{agb}, and its ultimate mass, i.e. the mass of the \ac{wd} remnant.

\begin{comment}

On the other hand, for masses greater than 15 M$_{\odot}$, mass-loss by stellar winds becomes important during all evolution phases, including the main sequence. Observations in the ultraviolet and infrared spectrum show that these luminous massive stars experience rapid mass outflows (stellar winds) that can gradually erode their outer layers. For masses greater than 30 M$_{\odot}$, the mass-loss rates, $\dot{M}$, are so great that the mass-loss timescale, $t_{ml} = M / \dot{M}$, becomes shorter than the nuclear timescale, $t_{nuc}$.
\end{comment}

\subsection{Convection}\label{sec:convection}

A variety of internal mixing process can impact the life cycle of stars. Apart from convection, convective overshooting, semi-convection, and rotationally induced mixing are the most important internal mixing process \citep{schootemeijer2019constraining} and are still poorly understood. In this work, semi-convection and rotationally induced mixing are not taken into account as these are expected to be more important for massive stars ($M>8$ M$_{\odot}$) \citep{langer2012presupernova}. However, convection and convective overshooting are being explored.

\begin{comment}
    
I investigate the role of convective overshooting, which proved to be mostly important for the inner high-density regions of the star. Convection, on the other hand, proved to play an important role in the construction of the hydrodynamical models (see \cref{sec:1D_to_3D}).
\end{comment}

Nuclear reactions at a star's core generate energy,
which is transported to the stellar surface, on roughly thermal timescale, see \cref{eq:thermal_timsecale}, preserving the \ac{te} of the star. As a result, an energy density gradient exists, resulting in a net energy flux towards the stellar surface. Since an energy density gradient is associated with a temperature gradient, the more the luminosity to be transported, the greater the temperature gradient required. However, there is an upper limit to the temperature gradient inside a star; This limit is called adiabatic temperature gradient,
\begin{equation}\label{eq:ad_tempe_grad}
    \nabla_{ad} = \left ( \frac{d logT}{d logP} \right)_{ad},
\end{equation}
and describes the logarithmic variation of temperature under adiabatic compression or expansion. An adiabatic process occurs on such a short (dynamical) timescale that there is no heat exchange with the environment.

In the case that the temperature gradient limit is surpassed, gas instability, called convection, occurs \citep{pols2011stellar}. Convection refers to the collective (bulk) cyclic movements of gas particles. This is a key activity in stellar interiors because it not only effectively transports energy, but also results in fast mixing. Another important aspect of convection is that the energy transportation and mixing occur on local dynamical timescale $t_{conv}$, where $t_{conv} << t_{th}$. This means that, while convection can efficiently transfer a large amount of energy, it does so at a nearly constant temperature $T$, indicating that the process is nearly adiabatic. Unfortunately though, it remains one of the least known aspects of stellar physics.

For such an adiabatic process the equation of state can be approximated by a polytropic relation
\begin{equation}\label{eq:adiabatic_eos}
    P \propto \rho^{\gamma_{ad}}.
\end{equation}
\cref{eq:adiabatic_eos} is of great importance. The absence of $T$ implies that the mechanical structure of convective stars or convective stellar envelopes is independent of the stellar temperature, (see \ac{rgb} in \cref{fig:HR_evolution}). For an ideal gas, ${\gamma_{ad}} = 5/3$ and/or $\nabla_{ad} = 0.4$ \citep{pols2011stellar}.

\begin{comment}
Another important consequence of \cref{eq:adiabatic_eos} is
\begin{equation}\label{}
    R \propto M^{-1/3}
\end{equation}
which derives by combining \cref{eq:adiabatic_eos} with the equations of 

It corresponds to a polytrope of $n = \frac{3}{2}$, where $\gamma_{ad} = 1+ \frac{1}{n}$ and 

From \cref{eq:ad_tempe_grad} one can see that the temperature stratification of a convective star can be described by a power law $T \propto P^{\nabla_{ad}}$.  Additionally, for an ideal gas, $P \propto \rho T$ and by combining the above one can easily see that

where $\gamma_{ad} = \frac{1}{1-\nabla_{ad}}$.
\end{comment}



{\bf Stability against convection}


Assuming that photons alone carry the produced energy, the dimensionless radiative temperature gradient is defined as
\begin{equation}\label{eq:radia_tempe_grad}
    \nabla_{rad} = \left ( \frac{d logT}{d logP} \right)_{rad} = \frac{3}{16 \pi \alpha c G} \frac{\kappa l P}{m T^4},
\end{equation}
where $\alpha$ is the radiation constant, $c$ the speed of light and $G$ the gravitational constant. The variable $l$ is the local luminosity, $\kappa$ the Rosseland mean opacity and $m$ the mass coordinate.  \cref{eq:radia_tempe_grad} describes the logarithmic variation of temperature with pressure, which represents the depth, for a star in \ac{he} if energy is transported only by radiation.

Using \cref{eq:ad_tempe_grad}, \cref{eq:radia_tempe_grad} and assuming ideal gas, I present the Ledoux criterion,  which states that a stellar layer is stable against convection if:
\begin{equation}\label{eq:Ledoux_criterion}
    \nabla_{rad} < \nabla_{rad} +  \nabla_{\mu},
\end{equation}
where $\nabla_{\mu}$ is the gradient of the mean molecular weight through the star. Because nuclear processes create more and more heavy elements in deeper layers, the mean molecular weight generally increases inwards. Typically, $\nabla_{\mu} \geq 0$, indicating that a composition gradient has a stabilizing effect.

For chemically homogeneous layers, the stability criterion against convection reduces to the Schwarzwild criterion:
\begin{equation}\label{eq:Schwarzwild_criterion}
    \nabla_{rad} < \nabla_{ad}.
\end{equation}
 

{\bf Convection occurrence}

Using \cref{eq:radia_tempe_grad} and the Schwarzwild criterion, \cref{eq:Schwarzwild_criterion}, convection occurs:
\begin{itemize}
    \item In opaque regions of the star, meaning $\kappa$ is large. Since the opacity increases with decreasing temperature \citep{pols2011stellar}, I expect the envelopes of cool star to be convective. For example, low-mass or intermediate-mass stars during the \ac{rgb} and \ac{agb} phase, see \cref{fig:HR_inter_stars} and \cref{fig:HR_evolution}.
    \item In regions of the star with large energy flux, meaning $\frac{l}{m}$ is large. For example, the cores of intermediate- and high-mass stars, where nuclear reactions are driven by the CNO-cycle ($\epsilon_{CNO} \propto \rho_c T_{c}^{18})$.
\end{itemize}

{\bf Convective overshooting}

Convective overshooting refers to the process of mixing beyond the boundaries of convective regions, which can occur when convective cells penetrate into radiative regions due to their non-zero velocity \citep{alongi1993evolutionary,brott2011rotating,schootemeijer2019constraining}. In stars with convective cores, e.g. intermediate- and high-mass stars, the size of the core is effectively enlarged through mechanisms such as convective core overshooting. This overshooting brings additional hydrogen into the core and therefore directly impacts the final He-core mass, while extents the \ac{ms} lifetime and affects the post-\ac{ms} evolution.

The extent of the overshooting region is not known reliably from theory. However, in stellar evolution calculations, overshooting is usually treated based on \cite{herwig2000evolution}. The latter investigate in detail the impact of overshooting on the evolution of \ac{agb} stars. Based on their model the extended mixing is treated time-dependently, and the efficiency decreases exponentially as the geometric distance from the convective boundary increases. 

