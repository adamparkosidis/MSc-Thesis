\section{Simulations' set up}

At this point, it is clear that I took many consecutive steps to model mass transfer in triple systems with a Roche-lobe filling outer star. A schematic representation of the entire process is provided in \cref{fig:schematic_method}
\begin{figure}[H]
    \centering
    \includegraphics[width=\textwidth]{Thesis/figures/method_schematic.pdf}
    \caption{Schematic representation of the steps taken to simulate mass transfer in the triple system with a Roche-lobe filling outer star. The white points are the purely gravitational particles representing the core and the inner binary components. The red points are the SPH particles representing the gas having an adaptive smoothing length.}
    \label{fig:schematic_method}
\end{figure}
In each step, I explored the importance of different parameters and made assumptions based on relative physics to better simulate the system's behavior. The outcome of a simulation is, of course, dependent on the simulation's setup; Furthermore, knowing the setup parameters is important for the reproducibility of the results. Hence, I summarize the codes' settings used for the final simulations for the reader.
\begin{table}[H]
    \centering
    \begin{tabular}{ |p{6.5cm}||p{6.5cm}|  }
     \hline
     \multicolumn{2}{|c|}{Simulation Settings} \\
     \hline
     MESA & GADGET-2 \\
     \hline
     Initial Masses = [3.2, 3.1. 5.5] M$_{\odot}$& $N_{particles}=50^4$ \\
     Solar metallicity& Self-gravity: True\\
     No winds& Adaptive smoothing length: True\\
     No overshooting& Softening = smoothing length\\
     No star rotation & Adiabatic EOS  \\
     Convection-occurrence (Schwarzwild) & Time step = $1/64 \times P_{in}$ \\
     Evolve until $\geq 1.1 \times R_{RLOF}$ & Artificial viscosity: $\alpha=0.5, \beta=1.0, \eta=0.1$ \\
     \hline
    \end{tabular}
        \centering
    \begin{tabular}{ |p{6.5cm}||p{6.5cm}|  }
     \hline
     \multicolumn{2}{|c|}{Simulation Settings} \\
     \hline
     Huayno & AMUSE \\
     \hline
     No softening length & Gravity-Hydrodynamics (2nd-order-coupling)\\
     Default time step ($<< 1/64 \times P_{in}$) &  Bridge time step = $1/64 \times P_{in}$\\
     \hline
    \end{tabular}
    \caption{ Various important settings and flags used for the different codes and AMUSE}
\label{tab:codes_settings}
\end{table}
