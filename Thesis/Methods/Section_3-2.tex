\section{Converting the 1D stellar evolution model to a 3D gas particle distribution}\label{sec:1D_to_3D}

In addition to fundamental parameters such as mass, radius, etc., MESA enables to access the stars internal structure. This information is essential to convert the 1D stellar models into 3D hydrodynamical realizations, providing a more comprehensive understanding of the physical processes involved in RLOF and the resulting mass transfer in these systems.

When the radius of the outer star exceeds its Roche limit, the 1D stellar evolution model is converted to a collection of SPH particles. This is accomplished by requesting the radial stellar structure profiles for density, temperature, mean molecular weight, and radius from the stellar evolution code. MESA divides a star into a series of spherical shells, which are represented by arrays in which these parameters are recorded. Following that, I produce a kinematically cold set of $N$ particles of mass $M_{ROLF}/N$ in a uniform spherical distribution. I now scale the particle locations radially to fit the density profile from the star up to its outer radius.

\subsection{Stellar Interior}\label{sub:core}

Because of the usage of equal-mass particles and the high concentration of stellar cores, the majority of the particles, and therefore the maximum resolution, will be in the stellar core, whereas the star's outer edge will be barely resolved. However, I want to investigate the hydrodynamical factors that dominate the star's outer layers in order to gain meaningful insights into the mass transfer process. To achieve that, I replace the stellar core with a single mass point, because the interior of the Roche lobe filling star barely affects the dynamics of the outer layers on the short-dynamical timescales associated with ROLF \citep{deupree2005structure}.  This technique not only provides higher resolution of the outer layers but also helps to circumvent computational run time constraints by using less particles.

The core particle's mass is unrelated to the mass of the hydrogen-exhausted stellar core, but rather a solution to the computational challenges of modeling big stars without changing the stellar envelope's behavior. Furthermore, it as a pure gravitational point mass with no pressure or internal energy. I investigate core masses corresponding to different percentages of the star's mass at the onset of ROLF, which are listed in \cref{tab:core_masses_ROLF}. 
\begin{table}[H]
    \centering
    \begin{tabular}{| c |}
       M$_{core}$  \\
       \hline
       10\% M$_{ROLF}$\\
       25\% M$_{ROLF}$\\
       50\% M$_{ROLF}$\\
       75\% M$_{ROLF}$
    \end{tabular}
    \caption{ Different core masses for the 3D hydrodynamical model of the tertiary at the beginning of ROLF}
    \label{tab:core_masses_ROLF}
\end{table}
Additionally, after investigating different number of particles, $N$, I adopt $N=50000$ due to the fact that higher $N$ numbers do not change the general behavior of the orbital parameters, while increase significantly the computational run time of the simulations.  Hence, each gas particle represents $M_{ROLF}/N = 11 \times 10^{-4}$ M$_{\odot}$.
\begin{figure}[H]
    \centering
    \includegraphics[width=0.9\textwidth]{Thesis/graphs/ROLF_density_profile.pdf}
    \caption{Radial density profile of the  $\xi$ Tau outer component at the onset of RLOF (blue line). MESA's 1D density profile of the tertiary is shown by the solid blue line. The shaded region represents 99.9\% of the star's enclosed mass. The dotted lines depict SPH models with varied core masses, M$_{core}$, which correspond to fractions of 0.1, 0.255, 0.5, and 0.75 of the overall mass of 5.5 M$_{\odot}$, while the larger points correspond to core particle densities, respectively.}
    \label{fig:stellar_density_ROLF}
\end{figure}
Lower core masses, M$_{core}$ $\leq$ 0.2M$_{ROLF}$, retain the problematic high density in the core, but larger core masses, M$_{core}$ $\geq$ 0.4 M$_{ROLF}$, cause the density in the envelope to depart significantly from the stellar structure model. This trend is apparent in \cref{fig:stellar_density_ROLF}. I use a M$_{core}$ = 0.25 M$_{ROLF}$ (= 1.4 M$_{\odot}$), which successfully overcomes the problematic high density inner stellar region and is in good agreement with envelopes' density profile.

\subsection{Stellar Envelope: Convective Case}\label{sub:envelope}

Following the distribution of particles based on the 1D density profile of the tertiary, I need to accurately specify important thermodynamic properties of the envelope (3D model), such as specific internal energy and specific entropy. Because the pressure of a gas is related to its internal energy, I need to examine the pressure sources in the envelope. In general, the pressure in a non-degenerate stellar interior will be given as 
\begin{equation}\label{eq:pressure}
    P(r) = \frac{1}{3} \alpha T(r)^4+ \frac{\mathcal{R}}{\mu(r)} \rho(r) T(r)
\end{equation}
where $\alpha$ and $\mathcal{R}$ are the radiation and universal gas constants, respectively. The first term in \cref{eq:pressure} is the radiation pressure exerted by photons and it will be dominant for massive hot stars, while it can be neglected for cool stars \citep{pols2011stellar}.

Although, I do not expect the radiative component to be dominant in a $5.5$ M$_{\odot}$ star, I verify this assumption. In \cref{fig:eos}, I present the boundaries between different sources of pressure. The coloured regions mark what pressure source dominate the equation of state in the considered regime of temperature and density, namely the radiation, the ideal gas or the (non-relativistic / relativistic) electron degeneracy
pressure.
\begin{figure}[H]
    \centering
    \includegraphics[width=0.9\textwidth]{Thesis/graphs/eos.pdf}
    \caption{ I create the graph using MESA \citep{paxton2010modules,paxton2013modules,paxton2015modules,paxton2019modules}.}
    \label{fig:eos}
\end{figure}
The blue line corresponds to the internal structure of the tertiary at moment of ROLF. Starting from right to left I follow the density and temperature of the star from the core to its surface. The black $x$ indicate, for decreasing $\rho$, the part of the star containing 1.375 M$_{\odot}$, 2.75 M$_{\odot}$, 4.125M$_{\odot}$ and 5.5 M$_{\odot}$, respectively. These values correspond to 25\%, 50\%, 75\% and 100\% of the star's enclosed mass.

It apparent from \cref{fig:eos} that at the moment of ROLF the pressure within the envelope is dominated by the ideal gas component, thus I can indeed neglect the radiative part of \eqref{eq:pressure}. Using
\begin{equation}\label{eq:pressure_ideal_gass}
    P(r) =  \frac{\mathcal{R}}{\mu(r)} \rho(r) T(r)
\end{equation}
each particle is allocated a unique specific internal energy based on the temperature and mean molecular mass profiles:
\begin{equation}\label{eq:internal_energy}
    u(r) = \frac{3}{2} \frac{k_B T(r)}{m \mu(r)}
\end{equation}
where $k_B$ is the Stefan-Boltzmann constant, $m$ the particle mass, and $T(r)$ and $\mu(r)$ the temperature and mean molecular mass profiles, respectively.

The particles at this point are kinematically cold meaning that the specific kinetic energies are zero and only their internal and potential energy contributes to the total energy of the gas. This is illustrated in \cref{fig:kinetic_internal_energies}, where I present 2D maps of the kinetic and internal particle energies, respectively.
\begin{figure}[H]
    \centering
    \begin{subfigure}{.5\textwidth}
    \centering
    \includegraphics[width=0.9\textwidth]{Thesis/graphs/tertiary_kin_energy_before_relaxation.pdf}
    \label{fig:mass_loss}
    \end{subfigure}%
    \begin{subfigure}{.5\textwidth}
    \centering
    \includegraphics[width=0.9\textwidth]{Thesis/graphs/tertiary_internal_energy_before_relaxation.pdf}
    \label{fig:radius_profile}
    \end{subfigure}
    \caption{ Kinetic and internal energies of the gas particles after converting the 1D stellar evolution model to a 3D gas particle distribution. The black point corresponds to the core particle which is a pure gravitational point mass with no pressure or internal energy.}
    \label{fig:kinetic_internal_energies}
\end{figure}

\begin{comment}
\begin{figure}[H]
    \centering
    \includegraphics[width=0.9\textwidth]{Thesis/graphs/tertiary_internal_energy_before_relaxation.pdf}
    \caption{Kinetic energies of the gas particles after converting the 1D stellar evolution model to a 3D gas particle distribution. The black point corresponds to the core particle which is  pure gravitational point mass with no pressure or internal energy.}
    \label{fig:kinetic_energies}
\end{figure}
\begin{figure}[H]
    \centering
    \includegraphics[width=0.9\textwidth]{Thesis/graphs/tertiary_kin_energy_before_relaxation.pdf}
    \caption{Kinetic energies of the gas particles after converting the 1D stellar evolution model to a 3D gas particle distribution. The black point corresponds to the core particle which is  pure gravitational point mass with no pressure or internal energy.}
    \label{fig:internal_energies}
\end{figure}
\end{comment}

Another important parameter of the gas is its specific entropy. It is often mathematically simpler, although not formally necessary, to define an entropic variable $A$. The entropic variable $A$ is closely linked (but not identical) to the specific entropy and it is defined as
\begin{equation}\label{eq:entropic_variable}
    A(r) = \frac{P(r)}{\rho(r)^{\gamma_{ad}}}
\end{equation}
where $\gamma_{ad}$ is the adiabatic index, and $P(r)$ and $\rho(r)$ the pressure and density profiles, respectively. 

The evaluation of the original entropic profile $A(r)$ of the stellar envelope is not trivial and one needs to consider the dominant pressure sources, see Eq. \eqref{eq:pressure}. Additionally, in a radiation-dominated gas, $\gamma_{ad} = 4/3$ is a more suitable adiabatic index (this is in general true when extremely relativistic particles dominate). For a mixture of gas and radiation both terms of \cref{eq:pressure} need to be considered with $4/3 \leq \gamma \leq 5/3$.

In \cref{fig:HR_ROLF}, I show that the tertiary is expected to overflow its Roche lobe during the early AGB phase. Hence, I expect that the envelope will be convective, see \cref{sub:HR_diagram}, meaning that not only the pressure in the envelope can be described by a polytrope, but $\gamma=5/3$, see \cref{sub:mixing}.

In order to verify that, I present a Kippenhahn diagram of the tertiary, see \cref{fig:kippen_plot}. This diagram displays the evolution of the structure of the star from the core to the photosphere. The vertical axis represents the position within the star in terms of enclosed mass and the horizontal axis represents the number of the models which are translated to the age of the star (hence the x-axis is not linear). In this diagram, one can see how different regions are discerned: The white parts are the radiative regions, the stripped parts represent the convective regions, and the variation of the color represent the
intensity of the energy generation at various regions. Darker color indicate that more energy is produced.
\begin{figure}[H]
    \centering
    \includegraphics[width=0.9\textwidth]{Thesis/graphs/jpg2pdf.pdf}
    \caption{Kippenhahn diagram of $\xi$ Tau outer component until the onset of RLOF. The vertical axis represents the position within the star in terms of enclosed mass and the horizontal axis represents the number of the models which are translated to the age of the star. The white parts are the radiative regions, the stripped parts represent the convective regions, and the variation of the color represent the intensity of the energy generation at various regions. Darker color indicate that more energy is produced. I create the graph using MESA \citep{paxton2010modules,paxton2013modules,paxton2015modules,paxton2019modules}.}
    \label{fig:kippen_plot}
\end{figure}

From \cref{fig:kippen_plot}, it is apparent that, at the moment of ROLF, the convective envelope has penetrated deep inside the star close to the actual core. Considering \eqref{eq:internal_energy} and \eqref{eq:entropic_variable}, I calculate the entropic profile $A(r)$ with $\gamma = 5/3$, such as:
\begin{equation}\label{eq:entropic_variable_2}
    A(r) = \frac{2}{3} u(r) \rho(r)^{-2/3}
\end{equation}



\subsection{Stellar Envelope: Radiative Case}\label{sub:envelope}

In the previous subsection I showed how I model $\xi$ Tau's outer component given its internal structure at the on set of ROLF. Furthermore, I highlighted the importance of radiation in stellar pressure profile for higher mass stars. Here, I present the some additional parameters that need to be considered in the case of a radiative envelope.

Low-mass, thus cold, stars have convective envelopes during their evolution, while massive stars with M $>16$M$_{\odot}$ remain with a radiative envelope pretty much until the end of their lives. Intermediate- and high-mass stars' envelopes (M $<16$M$_{\odot}$) can be either convective or radiative depending on the evolutionary phase in which ROLF occurs, see \cref{sec:single_star_evolution}. 

The density within a convective envelope falls of with pressure $\rho \propto P^{1/\gamma_{ad}}$, see Eq. \eqref{eq:adiabatic_eos}. If the envelope is stable against convection, the density gradient must be vary more steeply with pressure than for an adiabatic change by definition. This means that radiative envelopes are less dense in the outer layers and more centrally concentrated than convective envelopes \cite{pols2011stellar}. This is reflected in \cref{fig:density_profile_radiative}, where I plot the tertiary's radial density profile at $t=78.3$ Myr, when the star is in its helium-burning phase and has a fully radiative envelope, see \cref{fig:kippen_plot}.
\begin{figure}[H]
    \centering
    \includegraphics[width=0.9\textwidth]{Thesis/graphs/density_profile_radiative_envelope.pdf}
    \caption{Radial density profile of the  $\xi$ Tau outer component at $t=78.3$ Myr (blue line), when the star is in its helium-burning phase and has a fully radiative envelope. MESA's 1D density profile of the tertiary is shown by the solid blue line. The shaded region represents 99.9\% of the star's enclosed mass. The dotted lines depict SPH models with varied core masses, M$_{core}$, which correspond to fractions of 0.1, 0.255, 0.5, and 0.75 of the overall mass of 5.5 M$_{\odot}$, while the larger points correspond to core particle densities, respectively.}
    \label{fig:density_profile_radiative}
\end{figure}
Modeling the radiative envelope case is more challenging. From the computational point of view, the problem branches in to two sub-problems. First, a comparison between \cref{fig:stellar_density_ROLF} and \cref{{fig:density_profile_radiative}} immediately reveals the difference in scales. More specifically, in the radiative case someone needs to resolve 6-8 orders of magnitude in density. In \cref{fig:density_profile_radiative}, I use $N=10^6$ particles and still the outer layers of the star are not very well resolved. This is a direct consequence of the SPH method's nature and the use of equal mass particles. In general, SPH provides good resolution in high-density regions, however, only poorly resolves low-density regions. Second, using $N$ of 6 orders of magnitude, will result in extremely long computational time, as the CPU simulation time is not linear function of $N$, but rather the CPU time increases with the number of SPH particles as $NlogN$.

The problematic steepness of the density profile becomes more severe for high-mass stars. In conclusion, knowing the internal structure of the envelope at the moment of ROLF is critical for accurately modeling the stellar envelope. From a physics standpoint, someone must estimate the importance of radiation in order to accurately define the thermodynamic properties of the envelope using Eq. \eqref{eq:pressure}, \eqref{eq:entropic_variable} and a proper adiabatic index, $\gamma_{ad}$. 




\begin{comment}
    This parameter will be important in order to adjust correctly thermodynamic properties of the gas inside the softened zone.
\end{comment}
