\section{Stellar Evolution}

The Modules for Experiments in Stellar Astrophysics (MESA, \cite{paxton2010modules,paxton2013modules,paxton2015modules,paxton2019modules}) is an open-source 1D stellar evolution code used to model the evolution of stars from their birth to their death. The code combines many of the numerical and physics modules for simulations of a wide range of stellar evolution scenarios ranging from very low mass to massive stars, including advanced evolutionary phases, while solves the fully coupled structure and composition equations simultaneously. 

MESA allows me track to the evolution of the triple system components and obtain estimations of their properties at the moment of RLOF. In addition to fundamental parameters such as mass, radius, etc., MESA enables the access to the internal structure and detailed properties of the individual components. This information is essential for the conversion of one-dimensional stellar models into three-dimensional hydrodynamical realizations, providing a more comprehensive understanding of the physical processes involved in RLOF and the resulting mass transfer in these systems.

The stellar evolution calculations in this work are performed using the normal AMUSE parameters for MESA version r15140, with solar metallicity as the input. By the time the outer star approaches the radius of its Roche lobe, it has lost some mass and its radius is much bigger than when it was born. Table includes the important parameters of the system at the moment of RLOF. Figure 3 depicts the radial density profile of the outer component of the star at the moment of RLOF (green drawn line)



