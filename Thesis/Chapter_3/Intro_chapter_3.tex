\chapter{Simulations}

%\epigraph{The stars are not distant objects to be admired from afar. They are our partners in exploration and discovery, and we must learn to live and work with them if we are to achieve our goals.}{Arthur C. Clarke}

The evolution of outer Roche lobe overflow (RLOF) triple-star systems is influenced by various physical processes, including stellar evolution, gravitational dynamics, and hydrodynamics. I utilize the Astrophysical Multi-purpose Software Environment (AMUSE, \cite{portegies2018astrophysical}), a comprehensive computational tool, to accurately simulate and solve for these physical processes in a self-consistent manner. To model the evolution of the system star prior to outer stars's RLOF, I employ the stellar evolution code MESA \citep{paxton2010modules,paxton2013modules,paxton2015modules,paxton2019modules}. Once the outer star reached the stage where it approximately filled its Roche lobe, I pause the stellar evolution simulation and convert the one-dimensional stellar structure into a three-dimensional hydrodynamical model. This hydrodynamical model of the outer star is then relaxed and placed in orbit around the binary star. Subsequently, I monitor the intricate hydrodynamics of the mass transfer from the Roche lobe-filling outer star to the inner binary for multiple orbits, while simultaneously keeping track of the gravitational dynamics of the three stars and the hydrodynamics of the gas from the outer star. A schematic representation of the entire process is provided in 

\section{Stellar Evolution}

MESA (Modules for Experiments in Stellar Astrophysics, \cite{paxton2010modules,paxton2013modules,paxton2015modules,paxton2019modules}) is an open-source 1D stellar evolution code used to model the evolution of stars from their birth to their death. It is a Fortran code that combines many numerical and physics modules for simulations of a wide range of stellar evolution scenarios ranging from very low mass to massive stars, including advanced evolutionary phases. Key modules within MESA include the equation of state module, the nuclear reaction network module, and the hydrodynamic module. MESA also includes modules for convective and radiative energy transport, as well as modules for mass loss, rotation, and magnetic fields. 

The code's basic principle is that it solves the fully coupled structure and composition equations simultaneously. Thus, allows me to track the independent evolution of the triple system components and obtain estimations of their properties at the moment of RLOF. In addition to fundamental parameters such as mass, radius, etc., MESA enables the access to the internal structure and detailed properties of the individual components. This information is essential for the conversion of one-dimensional stellar models into three-dimensional hydrodynamical realizations, providing a more comprehensive understanding of the physical processes involved in RLOF and the resulting mass transfer in these systems.

The stellar evolution calculations in this work are performed using the normal AMUSE parameters for MESA version 2208, with solar metallicity as the input. By the time the outer star approaches the radius of its Roche lobe, it has lost some mass and its radius is much bigger than when it was born. Table includes the important parameters of the system at the moment of RLOF. Figure 3 depicts the radial density profile of the outer component of the star at the moment of RLOF (green drawn line)



