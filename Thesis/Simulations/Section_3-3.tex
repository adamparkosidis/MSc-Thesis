\section{Bringing the 3D hydrodynamical model in hydrostatic equilibrium}
 
Following the distribution of particles based on the 1D density profile of the tertiary, I need to accurately specify the thermodynamic properties of the envelope (3D model). Assuming ideal gas, each particle is allocated a unique specific internal energy based on the temperature and mean molecular mass profiles.
\begin{equation}\label{eq:internal_energy}
    u(r) = \frac{3}{2} \frac{k_B T(r)}{\mu(r)}
\end{equation}
where $k_B$ is the Stefan-Boltzmann constant, and $T(r)$ and $\mu(r)$ the temperature and mean molecular mass profiles, respectively. 

Since I consider the core particle to be a pure gravitational point mass with no pressure or internal energy, I need to avoid the star from collapsing on itself.
To achieve that, I use Plummer softening, $\epsilon$, to soften the core particle. Similar with GADGET-2, I adopt the conventional cubic spline of \cite{monaghan1985refined}, which drops to zero smoothly at $2.8 \epsilon$ (see Eq. \eqref{eq:spline_kernel}). The density and internal energy inside the softened zone are adjusted so that pressure equilibrium is maintained while the original entropic variable $A$ is preserved. The entropic variable is defined as
\begin{equation}
    A(r) = \frac{P(r)}{\rho(r)^{\gamma_{ad}}}
\end{equation}
where $\gamma$ is the adiabatic index, and $P(r)$ and $\rho(r)$ the pressure and density profiles, respectively. The entropic variable $A$ is closely linked (but not identical) to the specific entropy.

The motivation behind that is known as entropy sorting and it is derived by simulations of stellar collisions \citep{lombardi1995collisions,lombardi2003modelling,lombardi2006stellar,gaburov2008mixing,gaburov2010onset}, where both the entropic variable and the specific entropy are preserved in the absence of shocks and increase in their presence. To put it simply, the fluid with the highest entropy should be on top of the fluid with the lowest entropy in order to establish hydrodynamic stability. 

The evaluation of the original entropic profile $A(r)$ of the stellar envelope is not trivial and one needs to consider its internal structure. In general, the pressure in stellar interior will be given as 
\begin{equation}\label{eq:pressure}
    P(r) = \frac{1}{3} \alpha T(r)^4+ \frac{\mathcal{R}}{\mu(r)} \rho(r) T(r)
\end{equation}
where $\alpha$ and $\mathcal{R}$ are the radiation and universal gas constants, respectively. The first term in \cref{eq:pressure} is the radiation pressure exerted by photons and it will be dominant for massive hot stars, while it can be neglected for cool stars \citep{pols2011stellar}. Additionally, $\gamma_{ad} = 4/3$ for extremely relativistic particles (e.g. photons) \citep{pols2011stellar}. For a mixture of gas and radiation both terms of \cref{eq:pressure} need to be considered, while $4/3 \geq \gamma \leq 5/3$.

In Fig


I use an adiabatic EOS with $\gamma_{ad}$.




This process is implemented in the standard AMUSE routine {\it star\_to\_sph.py}. 










I consider an adiabatic EOS for state for the gas. This decision is predicated on the fact that the simulation run time is more than two orders of magnitude shorter than the $t_{th}$ of the tertiary, see \cref{tab:tertiary_timescale_ROLF}. Furthermore, I choose 





