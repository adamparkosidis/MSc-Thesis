\chapter{Simulations}\label{simulations}

\epigraph{The universe must be full of voices, calling from star to star in a myriad tongues. One day we shall join that cosmic conversation}{Arthur C. Clarke}

%\epigraph{The stars are not distant objects to be admired from afar. They are our partners in exploration and discovery, and we must learn to live and work with them if we are to achieve our goals.}{Arthur C. Clarke}

In order to model the evolution of outer RLOF triple-star systems, various physical processes, such as stellar evolution, gravitational dynamics, and hydrodynamics need to be considered. I utilize the Astrophysical Multi-purpose Software Environment (AMUSE, \cite{portegies2018astrophysical}), a comprehensive computational tool, to accurately simulate and solve for these physical processes in a self-consistent manner. To model the evolution of the system's stars prior to outer stars' RLOF, I employ the stellar evolution code MESA \citep{paxton2010modules,paxton2013modules,paxton2015modules,paxton2019modules}. Once the outer star reached the stage where it approximately filled its Roche lobe, I pause the stellar evolution simulation and convert the one-dimensional stellar structure into a three-dimensional hydrodynamical model. This hydrodynamical model of the outer star is then relaxed, using Gadget-2 \citep{springel2005cosmological}, and placed in orbit around the binary star. The inner binary components are seeing as point masses and their gravitational dynamics are handled by Huayno \citep{pelupessy2012n}. Subsequently, I monitor the intricate hydrodynamics of the mass transfer from the Roche lobe-filling outer star to the inner binary for multiple orbits, while simultaneously keeping track of the three stars gravitational dynamics and the outer star gas hydrodynamics. A schematic representation of the entire process is provided in 


\section{Stellar Evolution}\label{sec:stellar_evolution}

The stellar evolution calculations in this work are performed using the normal AMUSE parameters for MESA, with solar metallicity as the input.
MESA allows me to track the independent evolution of the triple system components and obtain estimations of their properties at the beginning of RLOF. In this work, I allow the tertiary to evolve until $R_{\star} = 1.1 \times R_{ROLF}$ which will be explained in this chapter along with the implications of that choice.

By the time the outer star approaches the radius of its Roche lobe, all three stars have lost some mass, while the tertiary's radius is much bigger than when it was formed. In order to accurately evaluate the Roche lobe sizes,  I need to estimate the masses of the stars at the ROLF moment. Unfortunately, $\xi$ Tau age is not known, but mass-loss via winds during the main sequence is expected to be unimportant for low- and intermediate- mass stars.

Despite that, I track the evolution of the tertiary's mass and radius profile. I utilize Reimer's \cite{reimers1975circumstellar} and Blocker's \cite{bloecker1995stellarI,bloecker1995stellarII} mass-loss prescriptions (see \cref{sec:single_star_evolution}) using commonly scaling factors of $\eta = 0.5$ and $\eta = 0.1$, respectivelly.

\begin{figure}[H]
    \centering
    \begin{subfigure}{.5\textwidth}
    \centering
    \includegraphics[width=0.9\textwidth]{Thesis/graphs/giant_1-1mass_loss.pdf}
    \label{fig:mass_loss}
    \end{subfigure}%
    \begin{subfigure}{.5\textwidth}
    \centering
    \includegraphics[width=0.9\textwidth]{Thesis/graphs/giant_1-1radius.pdf}
    \label{fig:radius_profile}
    \end{subfigure}
    \caption{ Mass and radius evolution of a 5.5 M$_{\odot}$ star at solar metallicity until ROLF. ZAMS and TAMS are noted by black circles, while bgininng and the end of helium burning by black squares. The stellar evolution models are made using MESA\citep{paxton2010modules,paxton2013modules,paxton2015modules,paxton2019modules}.}
\end{figure}

The tertiary loses less than $1\%$ of its 
initial mass, while the expected mass-loss from the binary components is even smaller, because less massive, and thus less luminus, stars will have weaker winds (see \cref{sec:single_star_evolution}). As a result, I do not correct for mass-loss and use the parameters listed in  \cref{tab:system_orbit_param}.

At this point, I also assume that the mass lost through winds has no effect on the stars. This assumption would be false in the case of more massive stars. As mass escapes from the star's surface, it carries angular momentum and can change the inner and outer orbital parameters. Furthermore, some of the escaped mass can be accreted, complicating the evolution of the two orbits even further. Given the small amount of mass loss, the aforementioned assumption is safe in this case.

In \cref{fig:HR_ROLF}, I present the evolution of the tertiaty on the HR diagram until the moment of ROLF. It is apparent that the tertiary will overflow its Roche lobe during the early AGB phase, after helium exhaustion, leading to a type C mass transfer.

\begin{figure}[H]
    \centering
    \includegraphics[width=0.9\textwidth]{Thesis/graphs/HR_1-1ROLF.pdf}
    \caption{Evolution of 5.5 M$_{\odot}$ at solar metallicity until the moment of ROLF. The track is calculated with MESA \citep{paxton2010modules,paxton2013modules,paxton2015modules,paxton2019modules}. The dashed lines show lines of constant radii by means of the Stefan–Boltzmann law.}
    \label{fig:HR_ROLF}
\end{figure}

In \cref{tab:system_orbit_param_ROLF} I summarize the important parameters of the system at the moment of RLOF. 

\begin{table}[H]
    \begin{adjustbox}{width=1\textwidth}
    \small
    \centering
    \begin{tabular}{| c c c c c c c c c c|}
       M$_1$ (M$_{\odot}$) & 
       M$_2$ (M$_{\odot}$) &
       M$_3$ (M$_{\odot}$) & $\alpha_{in}$ (au) &
       $\alpha_{out}$ (au) &
       $\epsilon_{in}$ &
       $\epsilon_{out}$ &
       $R_{ROLF}$ (au) &
       $t_{ROLF}$ (Myr) &
       $M_{ROLF}$  (M$_{\odot}$) \\
       \hline
       3.2 & 3.1 & 5.5 & 0.133 & 1.24 & 0.0 & 0.15 & 0.423 & 82.36 & 5.5
    \end{tabular}
    \end{adjustbox}
    \caption{ Orbital parameters of $\xi$ Tau system at the beginning of ROLF}
    \label{tab:system_orbit_param_ROLF}
\end{table}

Given these parameters, I also calculate the important timescales for the tertiary, which are listed in \cref{tab:tertiary_timescale_ROLF}.

\begin{table}[H]
    \centering
    \begin{tabular}{| c | c |}
       Timescale & Duration \\
       \hline
       $t_{dyn}$ & 4.85 day\\
       $t_{koz}$ & 16.81 yr\\
       $t_{th}$ & 4585.78 yr 
    \end{tabular}
    \caption{ Tertiary timescales at the beginning of ROLF.}
    \label{tab:tertiary_timescale_ROLF}
\end{table}





\section{Converting the 1D stellar evolution model to a 3D gas particle distribution}

In addition to fundamental parameters such as mass, radius, etc., MESA enables to access the stars internal structure. This information is essential to convert the 1D stellar models into 3D hydrodynamical realizations, providing a more comprehensive understanding of the physical processes involved in RLOF and the resulting mass transfer in these systems.

When the radius of the outer star exceeds its Roche limit, the 1D stellar evolution model is converted to a collection of SPH particles. This is accomplished by requesting the radial stellar structure profiles for density, temperature, mean molecular weight, and radius from the stellar evolution code. MESA divides a star into a series of spherical shells, which are represented by arrays in which these parameters are recorded. Following that, I produce a kinematically cold set of $N$ particles of mass $M_{ROLF}/N$ in a uniform spherical distribution. I now scale the particle locations radially to fit the density profile from the star up to its outer radius.

Because of the usage of equal-mass particles and the high concentration of stellar cores, the majority of the particles, and therefore the maximum resolution, will be in the stellar core, whereas the star's outer edge will be barely resolved. However, I want to investigate the hydrodynamical factors that dominate the star's outer layers in order to gain meaningful insights into the mass transfer process. To achieve that, I replace the stellar core with a single mass point, because the interior of the Roche lobe filling star barely affects the dynamics of the outer layers on the short-dynamical timescales associated with ROLF \citep{deupree2005structure}.  This technique not only provides higher resolution of the outer layers but also helps to circumvent computational run time constraints by using less particles.

The core particle's mass is unrelated to the mass of the hydrogen-exhausted stellar core, but rather a solution to the computational challenges of modeling big stars without changing the stellar envelope's behavior. Furthermore, it as a pure gravitational point mass with no pressure or internal energy. I investigate core masses corresponding to different percentages of the star's mass at the onset of ROLF, which are listed in \cref{tab:core_masses_ROLF}. 

\begin{table}[H]
    \centering
    \begin{tabular}{| c | c |}
       M$_{core}$  \\
       \hline
       10\% M$_{ROLF}$\\
       25\% M$_{ROLF}$\\
       50\% M$_{ROLF}$\\
       75\% M$_{ROLF}$
    \end{tabular}
    \caption{ Different core masses for the 3D hydrodynamical model of the tertiary at the beginning of ROLF}
    \label{tab:core_masses_ROLF}
\end{table}

Additionally, after investigating different number of particles, $N$, I adopt $N=50000$ due to the fact that higher $N$ numbers do not change the general behavior of the orbital parameters, while increase significantly the computational run time of the simulations.  Hence, each gas particle represents $M_{ROLF}/N = 11 \times 10^{-4}$ M$_{\odot}$.


\begin{figure}[H]
    \centering
    \includegraphics[width=0.9\textwidth]{Thesis/graphs/ROLF_density_profile.pdf}
    \caption{Radial density profile of the  $\xi$ Tau outer component at the onset of RLOF (blue line). MESA's 1D density profile of the tertiary is shown by the solid blue line. The shaded region represents 99.9\% of the star's enclosed mass. The dotted lines depict SPH models with varied core masses, M$_{core}$, which correspond to fractions of 0.1, 0.255, 0.5, and 0.75 of the overall mass of 5.5 M$_{\odot}$, while the larger points correspond to core particle densities, respectively.}
    \label{fig:stellar_density_ROLF}
\end{figure}

Lower core masses, M$_{core}$ $\leq$ 0.2M$_{ROLF}$, retain the problematic high density in the core, but larger core masses, M$_{core}$ $\geq$ 0.4 M$_{ROLF}$, cause the density in the envelope to depart significantly from the stellar structure model. This trend is apparent in \cref{fig:stellar_density_ROLF}. I use a M$_{core}$ = 0.255 M$_{ROLF}$ (= 1.4 M$_{\odot}$), which successfully overcomes the problematic high density inner stellar region and is in good agreement with envelopes' density profile.


\section{Bringing the 3D hydrodynamical model in hydrostatic equilibrium}
 
Following the distribution of particles based on the 1D density profile of the tertiary, I need to accurately specify the thermodynamic properties of the envelope (3D model). Assuming ideal gas, each particle is allocated a unique specific internal energy based on the temperature and mean molecular mass profiles.
\begin{equation}\label{eq:internal_energy}
    u(r) = \frac{3}{2} \frac{k_B T(r)}{\mu(r)}
\end{equation}
where $k_B$ is the Stefan-Boltzmann constant, and $T(r)$ and $\mu(r)$ the temperature and mean molecular mass profiles, respectively. 

Since I consider the core particle to be a pure gravitational point mass with no pressure or internal energy, I need to avoid the star from collapsing on itself.
To achieve that, I use Plummer softening, $\epsilon$, to soften the core particle. Similar with GADGET-2, I adopt the conventional cubic spline of \cite{monaghan1985refined}, which drops to zero smoothly at $2.8 \epsilon$ (see Eq. \eqref{eq:spline_kernel}). The density and internal energy inside the softened zone are adjusted so that pressure equilibrium is maintained while the original entropic variable $A$ is preserved. The entropic variable is defined as
\begin{equation}
    A(r) = \frac{P(r)}{\rho(r)^{\gamma_{ad}}}
\end{equation}
where $\gamma$ is the adiabatic index, and $P(r)$ and $\rho(r)$ the pressure and density profiles, respectively. The entropic variable $A$ is closely linked (but not identical) to the specific entropy.

The motivation behind that is known as entropy sorting and it is derived by simulations of stellar collisions \citep{lombardi1995collisions,lombardi2003modelling,lombardi2006stellar,gaburov2008mixing,gaburov2010onset}, where both the entropic variable and the specific entropy are preserved in the absence of shocks and increase in their presence. To put it simply, the fluid with the highest entropy should be on top of the fluid with the lowest entropy in order to establish hydrodynamic stability. 

The evaluation of the original entropic profile $A(r)$ of the stellar envelope is not trivial and one needs to consider its internal structure. In general, the pressure in stellar interior will be given as 
\begin{equation}\label{eq:pressure}
    P(r) = \frac{1}{3} \alpha T(r)^4+ \frac{\mathcal{R}}{\mu(r)} \rho(r) T(r)
\end{equation}
where $\alpha$ and $\mathcal{R}$ are the radiation and universal gas constants, respectively. The first term in \cref{eq:pressure} is the radiation pressure exerted by photons and it will be dominant for massive hot stars, while it can be neglected for cool stars \citep{pols2011stellar}. Additionally, $\gamma_{ad} = 4/3$ for extremely relativistic particles (e.g. photons) \citep{pols2011stellar}. For a mixture of gas and radiation both terms of \cref{eq:pressure} need to be considered, while $4/3 \geq \gamma \leq 5/3$.

In Fig


I use an adiabatic EOS with $\gamma_{ad}$.




This process is implemented in the standard AMUSE routine {\it star\_to\_sph.py}. 










I consider an adiabatic EOS for state for the gas. This decision is predicated on the fact that the simulation run time is more than two orders of magnitude shorter than the $t_{th}$ of the tertiary, see \cref{tab:tertiary_timescale_ROLF}. Furthermore, I choose 





