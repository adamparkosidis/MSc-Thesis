\section{Converting the 1D stellar evolution model to a 3D gas particle distribution}

In addition to fundamental parameters such as mass, radius, etc., MESA enables to access the stars internal structure. This information is essential to convert the 1D stellar models into 3D hydrodynamical realizations, providing a more comprehensive understanding of the physical processes involved in RLOF and the resulting mass transfer in these systems.

When the radius of the outer star exceeds its Roche limit, the 1D stellar evolution model is converted to a collection of SPH particles. This is accomplished by requesting the radial stellar structure profiles for density, temperature, mean molecular weight, and radius from the stellar evolution code. MESA divides a star into a series of spherical shells, which are represented by arrays in which these parameters are recorded. Following that, I produce a kinematically cold set of $N$ particles of mass $M_{ROLF}/N$ in a uniform spherical distribution. I now scale the particle locations radially to fit the density profile from the star up to its outer radius.

Because of the usage of equal-mass particles and the high concentration of stellar cores, the majority of the particles, and therefore the maximum resolution, will be in the stellar core, whereas the star's outer edge will be barely resolved. However, I want to investigate the hydrodynamical factors that dominate the star's outer layers in order to gain meaningful insights into the mass transfer process. To achieve that, I replace the stellar core with a single mass point, because the interior of the Roche lobe filling star barely affects the dynamics of the outer layers on the short-dynamical timescales associated with ROLF \citep{deupree2005structure}.  This technique not only provides higher resolution of the outer layers but also helps to circumvent computational run time constraints by using less particles.

The core particle's mass is unrelated to the mass of the hydrogen-exhausted stellar core, but rather a solution to the computational challenges of modeling big stars without changing the stellar envelope's behavior. Furthermore, it as a pure gravitational point mass with no pressure or internal energy. I investigate core masses corresponding to different percentages of the star's mass at the onset of ROLF, which are listed in \cref{tab:core_masses_ROLF}. 

\begin{table}[H]
    \centering
    \begin{tabular}{| c | c |}
       M$_{core}$  \\
       \hline
       10\% M$_{ROLF}$\\
       25\% M$_{ROLF}$\\
       50\% M$_{ROLF}$\\
       75\% M$_{ROLF}$
    \end{tabular}
    \caption{ Different core masses for the 3D hydrodynamical model of the tertiary at the beginning of ROLF}
    \label{tab:core_masses_ROLF}
\end{table}

Additionally, after investigating different number of particles, $N$, I adopt $N=50000$ due to the fact that higher $N$ numbers do not change the general behavior of the orbital parameters, while increase significantly the computational run time of the simulations.  Hence, each gas particle represents $M_{ROLF}/N = 11 \times 10^{-4}$ M$_{\odot}$.


\begin{figure}[H]
    \centering
    \includegraphics[width=0.9\textwidth]{Thesis/graphs/ROLF_density_profile.pdf}
    \caption{Radial density profile of the  $\xi$ Tau outer component at the onset of RLOF (blue line). MESA's 1D density profile of the tertiary is shown by the solid blue line. The shaded region represents 99.9\% of the star's enclosed mass. The dotted lines depict SPH models with varied core masses, M$_{core}$, which correspond to fractions of 0.1, 0.255, 0.5, and 0.75 of the overall mass of 5.5 M$_{\odot}$, while the larger points correspond to core particle densities, respectively.}
    \label{fig:stellar_density_ROLF}
\end{figure}

Lower core masses, M$_{core}$ $\leq$ 0.2M$_{ROLF}$, retain the problematic high density in the core, but larger core masses, M$_{core}$ $\geq$ 0.4 M$_{ROLF}$, cause the density in the envelope to depart significantly from the stellar structure model. This trend is apparent in \cref{fig:stellar_density_ROLF}. I use a M$_{core}$ = 0.255 M$_{ROLF}$ (= 1.4 M$_{\odot}$), which successfully overcomes the problematic high density inner stellar region and is in good agreement with envelopes' density profile.

