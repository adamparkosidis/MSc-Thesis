\chapter{Introduction}

\epigraph{The universe must be full of voices, calling from star to star in a myriad tongues. One day we shall join that cosmic conversation}{Arthur C. Clarke}

Stellar formation and evolution is one of the oldest and most studied fields in astrophysics. In their attempt to achieve hydrostatic and thermal equilibrium, stars generate temperatures and pressures that allow for nuclear burning, and thus the emission of the starlight that we see. The cycles of nuclear burning and fuel exhaustion regulate the evolution of a star and set the various phases during the stellar lifetime.

The evolution of a star in isolation, namely single star evolution, is predominantly determined by the stellar mass. As a result, stars with $M \leq 2$ M$_{\odot}$, $2 < M \leq 8$ M$_{\odot}$ and $M > 8$ M$_{\odot}$ are classified as low-, intermediate-, and high-mass, respectively. Furthermore, the evolution is only slightly affected by the initial chemical composition.  On the other hand, a number of internal mixing process can significantly alter the evolution, particularly of massive ($M>8$ M$_{\odot}$), stars \citep{langer2012presupernova}. Core overshooting, semiconvection, and rotationally induced mixing are the most important internal mixing process \citep{schootemeijer2019constraining} and are still poorly understood. 


\section{Evolution of Massive stars}

\section{Hierarchical Triple Star Systems}



Although the fundamentals of single star and binary evolution have long been acknowledged \citep{postnov2014evolution,toonen2014popcorn}, the long-term evolution of stellar triples remains unknown. In the simple case, stable triple systems, which are hierarchical, namely, consist of an inner and an outer binary orbit, i.e., the tertiary (third object). The secular evolution of such systems is the modification of orbital elements over timescales substantially larger than the system's dynamical timescale. Hence, the presence of the outer star has no influence on the history of the inner binary and the evolution of the inner binary and the tertiary can be discussed independently. In other cases, a third star in orbit around a binary system can drastically influence the development the system via interactions, which influence the orbital elements of the inner and outer orbit through changes in energy and angular momentum. Consequently, triple stellar evolution processes which are unique to systems with multiplicities of higher orders than binaries can alter stable systems to become unstable.