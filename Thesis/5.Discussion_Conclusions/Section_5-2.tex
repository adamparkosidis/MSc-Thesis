\section{Conclusions}

Mass transfer in hierarchical triple systems, and more particularly, RLOF by the outer star, is likely to be very different from mass transfer in regular binary systems. The process is fairly complicated, but a qualitative picture is the following: In the presence of a donor with a convective envelope and the absence of a circumbinary disk, mass transfer is expected to be highly non-conservative. As a result, the outer orbit shrinks, while tidal interactions between the inner and outer orbits circularize the latter and push it toward coplanarity with the inner orbit. Given the current resolution, the response of the inner orbit is unclear. The orbit is expected to shrink as a result of the formation of a common envelope-like structure around the inner binary, while higher angles of the incoming mass stream result in faster decay of the orbit. Nonetheless, higher resolution simulations are required to confirm the above trend. Finally, the effect of the mass transfer on the eccentricity of the inner orbit is negligible. However, mutual inclinations well inside the Lidov-Kozai regime result to exchange of angular momentum between the two orbits. The latter appear in the form of periodic variations of the inner eccentricity and the mutual inclination.

The accretion efficiency and the amount of angular momentum carried away by ejected matter are critical for a quantitative description of the process. Assuming that the ejected mass carries a constant amount of specific angular momentum, the lower the accretion efficiency, $\beta$, of the inner binary, the greater the decay of the outer orbit, see \cref{fig:comparison_analytical_model_max}. Consequently, the aforementioned rates of change of orbital parameters are also higher. Such systems are expected become dynamically unstable in shorter timescales. Close encounters and collisions are expected to enhance, which tend to dissolve systems to lower orders \citep{van2007formation}. Systems with relatively higher $\beta$ values are expected to be relatively more stable as less angular momentum is lost, see \cref{fig:comparison_analytical_model_max}. The mass transfer is expected to be relatively more stable and to occur during longer timescales.

The higher accretion efficiency of material during longer timescales allow the inner binary stars to increase their masses and rotational velocities \cite{packet1981rotation}. Consequently, their evolutionary paths may be altered possibly leading to the formation of exotic objects, such as blue strugglers, and/or rapidly rotated metal-rich binary stars with somehow more homogeneous envelopes. Even in the case of stellar collisions, the final products will result from now more massive and evolved progenitors. Thus the merger outcome between the two cases (low and high $\beta$) may differ in mass and metallicity. In conclusion, combining theoretical and observational studies is essential in further constraining $\beta$, $\eta$ and generally the possible evolutionary outcomes of mass transfer in triples.



