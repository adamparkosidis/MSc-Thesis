\section{Conclusions}

Mass transfer in hierarchical triple systems, and more particularly, RLOF by the outer star, is different from mass transfer in regular binary systems. The process is fairly complicated, but a qualitative picture is the following: In the presence of a donor with a convective envelope and the absence of a circumbinary disk, mass transfer is expected to be highly non-conservative. As a result, the outer orbit shrinks, while tidal interactions between the inner and outer orbits circularize the latter and push it toward coplanarity with the inner orbit. Given the current resolution, the response of the inner orbit is unclear. The orbit widens for low angles of the incoming mass stream, while shrinks for higher ones. Nonetheless, higher resolution simulations are required to confirm the above trend. Finally, the effect of the mass transfer on the eccentricity of the inner orbit is negligible. However, mutual inclinations well inside the Lidov-Kozai regime result to exchange of angular momentum between the two orbits. The latter appear in the form of periodic variations of the inner eccentricity and the mutual inclination.

The accretion efficiency and the amount of angular momentum carried away by ejected matter are critical for a quantitative description of the process. Assuming that the ejected mass carries a constant amount of specific angular momentum, the lower the accretion efficiency, $\beta$, of the inner binary, the greater the decay of the outer orbit, see \cref{fig:comparison_analytical_model_max}. Consequently, the aforementioned rates of change of orbital parameters are also higher. Such systems are expected become dynamically unstable in shorter timescales. Close encounters and collisions are expected to enhance, which tend to dissolve systems to lower orders \citep{van2007formation}. Systems with higher $\beta$ values are expected to be relatively more stable as less angular momentum is lost during the same time period, see \cref{fig:comparison_analytical_model_max}. The mass transfer is expected to be relatively more stable, at least during the early phase of the process, and to span during longer timescales.

The higher accretion efficiency may also have some observational implications. Because more mass is accreted and also during longer timescales, the inner binary stars can increase their masses and rotational velocities \citep{packet1981rotation}. Consequently, their evolutionary paths may be altered possibly leading to the formation of exotic objects, such as blue strugglers \citep{zwart2019triple}, and/or rapidly rotated binary stars with somehow more homogeneous envelopes \citep{dorozsmai2023stellar}. Furthermore, convective mixing in the deep convective envelope, see \cref{fig:kippen_plot}, brings heavier elements to the donor's surface, see \cref{sec:convection}, and this is the material that is transferred via \ac{rlof}. Hence, effective accretion can also lead to metal-rich binary stars. Even in the case of stellar collisions, the final products will result from now more massive and evolved progenitors. Thus the merger outcome between the two cases (low and high $\beta$) may differ in mass and metallicity. For example, \cite{gao2023stellar} proposed that Barium stars can be formed in hierarchical triples, where \ac{rlof} by the outer star is followed by a merger of the barium rich inner binary components. In conclusion, combining theoretical and observational studies is essential in further constraining $\beta$, $\eta$ and the possible evolutionary outcomes of mass-transferring triples.



