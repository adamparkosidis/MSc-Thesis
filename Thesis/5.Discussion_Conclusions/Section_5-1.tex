\section{Summary}

In this work I present one of the few studies of simulating mass transfer in hierarchical triple systems with a focus on RLOF by the tertiary towards the inner binary. 

In the first part, I offer a detailed analysis of the process of creating hydrodynamical models representing a star using SPH particles, while accounting for its internal structure. In particular, I examine both convective and radiative envelopes, highlighting the various physical mechanisms that must be considered when defining their thermodynamic properties, respectively. In addition, by delving into the details of the relaxation process, I provide useful guidelines for creating stable hydrodynamical models. Finally, I demonstrate the method's efficiency in modeling low- and intermediate-mass stars with convective envelopes, as well as its shortcomings in modeling radiative envelopes, particularly those of massive stars.

In the second part, I describe qualitatively the effect of mass transfer in the evolution of the inner and outer orbits. This is a complicated hydrodynamical problem. On the one hand, the amount of mass and angular momentum that is lost from the system is critical for the system's evolution. On the other hand, there are nearly no theoretical predictions for the aforementioned quantities. Here, I provide some estimates and qualitative comparisons for two cases: In one case the accretion radii of the inner binary components are slightly smaller than the Roche lobe radii of their initial configuration, see \cref{fig:triple_equop}. In this scenario, I approach the maximal accretion case, effectively setting a lower limit to the amount of lost mass from the system. In the other case, the accretion radii correspond to the physical radii of the binary components, essentially illustrating the scenario of minimal accretion setting an upper limit to the amount of lost mass from the system. Consequently, I evaluate the importance of the inner binary's accretion efficiency and provide qualitative description of the system's response to the mass transfer in the above cases. Furthermore, I investigate the effect of the initial inclination of the outer orbit relative to the inner orbit, a parameter that is expected to be critical for the mass transfer process. Despite the fact that my simulations' resolution is low, I extract valuable information about the qualitative behavior of the system in these different cases. Finally, I demonstrate that the coupled solver that I deploy can properly capture the intricate three-body dynamics by accommodating the details of Lidov-Kozai cycles. 

The acquired knowledge in hydrodynamically modeling stars, combined with the successful capture of the intricate three-body dynamics, is perhaps the most important indirect result of this work. The code developed for this project is highly adaptable, and minor adjustments allow for the investigation of mass transfer via \ac{rlof} in various triple configurations. For example, the 3D hydrodynamical model could replace one of the inner binary stars effectively simulating \ac{rl} in the inner binary and also include dynamical perturbations due to an outer star. 

