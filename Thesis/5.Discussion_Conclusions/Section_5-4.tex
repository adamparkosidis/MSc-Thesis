\section{Future of $\xi$ Tau}\label{sec:discussion}

It is not easy to predict the system's future. One critical parameter is the donor star's reaction to mass-loss. On the one hand, the reduction of the donor's mass shrink its Roche lobe. On the other hand, the donor is a red giant with a deep convective envelope. The convective envelope ensures that the radius of the giant remains constant or even expands adiabatically due to mass loss during the early stage of the process, see Onno Pols `Binary Stars`. At the same time, the outer orbit becomes circular and coplanar with the inner orbit. As a result, a continuous and accelerated mass-loss in a circular orbit, is expected to initially replace the current picture of periodic mass-loss. To draw conclusions about later stages, detailed calculations of the donor's Roche lobe and adiabatic response to mass-loss are required. If the donor's adiabatic response is unable to keep it within its Roche lobe, the result is an ever-increasing mass-transfer rate, resulting in dynamically unstable mass transfer. In the opposite case, after significant mass-loss, the donor will react adiabatically and detach from its Roche lobe. The radius will continue to expand on a thermal timescale until \ac{rlof} is re-established, resulting in a thermal-timescale mass transfer, which is self-regulated.

The second critical parameter is the response of the inner orbit. If the inner orbit truly expands, the system enters the unstable regime, see \cref{eq:stability_regime}, and most likely will dissolve to a lower order via collision or ejection of one star. If the inner orbit shrinks, though, the ratio between the semi-major axes' rate of change,
\begin{equation}
    \frac{\dot{a_{in}}/a_{in}}{\dot{a_{out}}/a_{out}},
\end{equation}
determines the system's future. In this case, the system shrinks as a whole and if the ratio is close to unity, the system remains dynamically stable. Furthermore, if $\dot{a_{in}}/a_{in}$ is significant enough, the system will enter in a new phase of \ac{rlof} initiating in the inner binary, and thus a new cycle of interesting triple evolution will take place. Despite, the multiple scenarios, observations of hierarchical triple systems with white dwarfs orbiting close \ac{ms} binaries, where both orbits are circular and on the same plane, would be attractive candidates for the evolutionary scenario outlined in this work.

\begin{comment}
    

For complicity, I mention that in the aforementioned case where the system remains dynamically stable and the inner binary accretes matter some extra parameters needs to be considered. These are the response of the binary components to the accretion. Accretion is likely to spin-up the accreting stars to high rotational velocities \citep{packet1981rotation} setting an upper limit to their individual accretion efficiency.


The evolution of binary systems during mass transfer can be quite complicated, and the addition of a third star adds to the process's complexity. The late type B case of \ac{rlof} by an outer star toward an inner binary presented here is short-lived, probably occurring on the donor's thermal timescale, $\Delta t \approx t_{th}$, or on $t_{dyn} \leq \Delta t \leq t_{th}$. Hence, direct observational counterparts are expected to be extremely rare.  However, considering the implications of the accretion efficiency and angular momentum loss some further theoretical estimations can be made. 

On the one hand, low values of $\beta$ should be connected with more unstable systems. As angular momentum is lost rapidly, the fraction of the inner and outer orbit changes in shorter timescales, see \cref{eq:stability_regime}. Hence, such systems enter the unstable regime in shorter timescales. Close encounters and collisions are expected to enhance, which tend to dissolve systems to lower orders \citep{van2007formation}. On the other hand, higher values of $\beta$ are connected with more stable systems as less amount of angular momentum is lost assuming the same $\eta$, see \cref{fig:comparison_analytical_model_max}. The mass transfer is expected to be somehow more stable and to occur during longer timescales. The higher accretion efficiency of material during longer timescales allow the inner binary stars to increase their masses and rotational velocities. The latter will impact their evolutionary paths and can possibly lead to the formation of exotic objects, such as blue strugglers, and/or rapidly rotated stars with somehow more homogeneous envelopes. 

More specifically, peculiar observed systems could possibly reveal insights regarding their progenitors 


\subsection{Future work}

The evolution of binary systems during mass transfer can be quite complicated, and the addition of a third star adds to the process's complexity. Additional constraints on the accretion efficiency and angular momentum loss are critical in order to better understand the evolution and the outcome of these systems. First, it is important to repeat the simulations with higher resolution in order to verify the observed trends and the calculated values for $\beta$ and $\eta$. 
\end{comment}






