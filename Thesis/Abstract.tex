\chapter*{Abstract}

Mass transfer in hierarchical triple systems, and more specifically, \ac{rlof} by the outer star, is expected to be considerably different from mass transfer in an ordinary binary system. The mass cannot be simply accreted by the inner objects, but may instead form a circumbinary disk or be expelled via a slingshot effect due to the inner binary rotation. Stellar evolution, gravitational dynamics, and hydrodynamics all play important roles in the process. In the first part, I create 3D hydrodynamical models of post main-sequence stars based on detailed 1D stellar evolution models. In the second part, I use AMUSE to couple hydrodynamics with a high accuracy gravitational integrator and solve these physical processes in a self-consistent  manner. Hence, I simulate the phase of mass transfer in a hierarchical triple system in which the tertiary star will overfill its Roche lobe before any of the inner stars leave the main sequence. I encounter a fairly non-conservative mass transfer, and while I quantify its impact on the inner and outer orbits, predicting the end of the mass transfer phase and the appearance of the resulting system is difficult. However, I provide some preliminary estimations of the system's accretion efficiency and the amount of angular momentum lost. Finally, I compare my results with other studies and speculate that the formation of a circumbinary disk around the inner binary probably leads to significantly more conservative mass transfer. 




