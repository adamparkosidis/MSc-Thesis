\documentclass{uva-inf-article}
\usepackage{float}
\usepackage{amsmath}
\usepackage{subcaption}
\numberwithin{equation}{section}
\usepackage{hyperref}
\hypersetup{
    colorlinks,
    linkcolor={blue!100!black},
    citecolor={blue!100!black},
    urlcolor={blue!100!black}
}
\usepackage[noabbrev,capitalize,nameinlink]{cleveref}
\urlstyle{same}

\usepackage{natbib}


\title{Project Proposal}

\assignment{Research Project Physics and Astronomy}
\assignmenttype{Report}


\authors{Adam Parkosidis}
\uvanetids{13950742}

\tutor{}
\mentor{Silvia Toonen}
\docent{}

\course{}
\courseid{5354RPP60Y}

\date{\today}


\begin{document}
\maketitle


\section{Introduction}

Despite the inherent rarity predicted by the initial mass function (IMF, see e.g. \cite{chabrier2005initial, dib2018emergence}), massive stars ($M \geq 8M_{\odot}$) play a key role in the evolution of the Universe. They are the main source of UV radiation and heavy elements. They serve as a significant source of mixing and turbulence in the interstellar medium (ISM) of galaxies through a combination of winds, outflows, expanding HII regions, and supernova explosions. Galactic dynamos are powered by turbulence in conjunction with differential rotation. Cosmic rays are accelerated by the interaction of galactic magnetic fields and supernova shock fronts. The ISM is primarily heated by cosmic rays, UV radiation, and the dissipation of turbulence, whereas it is finally cooled by heavy metals present in dust, molecules, and in atomic/ionic form. Therefore, massive stars have a significant impact on galaxies' physical, chemical, and morphological structure \citep{kennicutt2005role}. However, the physical mechanisms behind the birth, development, and demise of massive stars remain elusive in comparison with low-mass stars \citep{zinnecker2007toward}. 

The fraction of systems with companions grows with mass \citep{moe2017mind}, as a result massive stars are seldomly formed in isolation. In contrast, most of massive stars are created in binary or higher order multiple systems with $\sim 50\%$ of spectral type B stars be in triples \citep{moe2017mind}. Stellar evolution in multiples in combination with gravitational interactions, angular momentum exchange, mass transfer, intense outflows and winds, ionizing radiation, and supernovae affect the evolution of these systems and can result into exotic end-products \citep{sana2012binary, toonen2016evolution}. However, in many of these relatively unusual objects that are difficult to be explained by binary evolution, triples provide promising evolutionary pathways e.g. blue straglers \citet{winn2009spin}) Thus, a detailed examination of the evolution of triples demands a self consistent treatment of three-body dynamics and stellar evolution. 

A very interesting system is TIC 470710327. This massive compact hierarchical triple system, first discovered in data provided by the Transiting Exoplanet Survey Satellite (TESS) and confirmed with the HERMES spectrograph, was published by \cite{eisner2022planet}. TIC 470710327 is made up of a massive main-sequence (MS) star on an orbit of $52.04$ days and an inner circular binary of main-sequence stars with a period of $1.10$ days.  The tertiary star has a mass of $14.5–16 M_{\odot}$, while the individual masses of the inner binary estimated of $6-7 M_{\odot}$ and $5.5-6.3 M_{\odot}$ respectively. The tertiary's orbital arrangement is more complicated, with an eccentricity of $e = 0.3$ and a mutual inclination of $i = 16.8^{+4.2}_{-1.4} \deg$. Such a configuration of masses is relatively rare \citep{de2014evolution} and TIC 470710327's genesis is still an open question. One would anticipate that the more massive secondary star was the first to emerge since more massive stars have shorter Kelvin-Helmholtz timeframes than stars with lower masses. \cite{vigna2022mergers} presented a progenitor scenario for TIC 470710327 in which $2+2$ quadruple dynamics resulted in Zeipel–Lidov–Kozai (ZKL) oscillations triggering the merger of the more massive binary either during late phases of star formation or several $Myr$ after the zero-age main sequence (ZAMS)

Apart from the genesis of the system, also its future evolution is unknown. The measured mass ratio suggests that the $14–17 M_{\odot}$ tertiary star, which will be the first star to develop off the MS, will be the one to drive the system's further evolution. It is predicted to fill its Roche lobe and begin transferring mass to the inner binary. Although is currently dynamically stable, the outcome of such process is a strictly hydrodynamical problem \citep{de2014evolution} and depends on the nature of the mass transfer and the response of the inner binary. If the compactness of the inner binary is high enough a circumbinary disk may form. According to \cite{leigh2020mergers}, such a situation favors evolution toward equal mass inner binary stars by causing preferential accretion to the lowest mass component. On the other hand, friction may cause the inner orbit to shrink if the mass transfer stream crosses it, which might result in a contact system and/or a merger \citep{de2014evolution}. Due to the merger's rejuvenating effects as well as the accretion from the tertiary star, such a merger remnant may be regarded as a blue straggler. In the case of a merger, the triple will be reduced to a new binary system opening a new discussion about the possible fate of the system.

In this project, we propose the massive compact hierarchical triple system TIC 470710327 as our research target. We will use the Astrophysical Multipurpose Software Environment (AMUSE, \cite{portegies2018astrophysical}) to simulate the evolution the system and try to predict its future. According to the Massive Star Catalog, this kind of mass transfer—from an outer star to an inner binary—should take place in $ \sim 1\%$ of all triple systems \citep{de2014evolution,hamers2022statistical}. Hence, the examination of this extraordinary system promises new constraints on both the formation scenarios of multiple massive-star systems as well as their exotic evolutionary end-products.
\\

\section{Project Goal}

Predict the future of the massive, compact, hierarchical triple system TIC 470710327 and put constraints on both the formation scenarios of multiple massive star systems as well as their exotic evolutionary end-products.

\section{Project scope}
\underline{Within scope:}
\begin{itemize}
    \item Understand the physics behind the interaction of stars in triple systems.
    \item Understand the stellar evolution of single massive stars in the mass regime of interest.
    \item Become familiar with AMUSE framework.
    \item Present the project in the API.
    \item Deliver a written Thesis.
\end{itemize}

\hspace*{-0.7cm}\underline{Out of scope:}
\begin{itemize}
    \item Further develop the code to extend its usage.
\end{itemize}

\section{Project Tasks}

\underline{01/09/2022 - 31/10/2022}:
\begin{itemize}
    \item Read papers regarding the physics behind the interaction of stars in triple systems.
    \item Read papers regarding the stellar evolution of single massive stars in the mass regime of interest. 
    \item Complete the interactive tutorial of the AMUSE framework.
\end{itemize}

\hspace*{-0.7cm}\underline{01/11/2022 - 22/12/2022}:
\begin{itemize}
    \item Become familiar with AMUSE framework by running test simualtions
    \item Reproduce the results of \citep{de2014evolution} using AMUSE in order to test the code and further understand the tertiary driven mass transfer.
    \item Deliver the first draft of the Thesis introduction to my mentor.
\end{itemize}

\hspace*{-0.7cm}\underline{10/01/2023 - 10/03/2023}:
\begin{itemize}
    \item Alter the code routines based on my scientific goal/objectives.
    \item  Run a complete simulation with the system parameters.
    \item  Analyze the results and find out which free parameters affect more the evolution of the system and possibly its end-product.
    \item Write down clearly the analysis of the results.
\end{itemize}

\hspace*{-0.7cm}\underline{11/03/2023 - 31/05/2023}:
\begin{itemize}
    \item Test a range of values for the most important free parameters. Consider observations or/and theoretical models in order to constrain the range of values.
    \item Further analyze how these free parameters affect the evolution of the system and possibly its end-product.
    \item Compare the results with results from similar studies.
    \item  Write down clearly the analysis of the results.
\end{itemize}

\hspace*{-0.7cm}\underline{01/06/2023 - 30/06/2023}:
\begin{itemize}
    \item Write down the Thesis.
    \item Give the first complete draft for corrections.
    \item Prepare a presentation of the project .
\end{itemize}

\hspace*{-0.7cm}\underline{30/06/2023 - 31/07/2023}:
\begin{itemize}
    \item Take into account the corrections of the mentor and apply them.
    \item Read the Thesis and correct any grammatical error etc.
    \item Submit the final version of the Thesis.
\end{itemize}

\subsection{Project Deliverables}
\underline{The tangible or intangible items produced by the project through out the academic year 2022-2023:}\\
\vspace{0.1cm}
\begin{itemize}
    \item A report which reproduces the results of {\it de Vries et al. 2014}.
    \item The first draft of the Thesis introduction.
    \item Figures analysing the data of the simulations.
    \item An evolution scenario (or more than one) for the target system.
    \item A presentation of the Thesis.
    \item The final version of the Thesis.
\end{itemize}


\bibliographystyle{plainnat}
\bibliography{references}
\end{document}


